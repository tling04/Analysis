\documentclass[12pt, letter]{article}
\usepackage[utf8]{inputenc}
\usepackage[a4paper, total={6in, 8in}]{geometry}
\usepackage{tikz}
\usepackage[T1]{fontenc}
\usepackage{listings}
\usepackage{graphicx}
\usepackage{amsfonts}
\usepackage{amsmath}
\usepackage{amssymb}
\usepackage{amsthm}
\usepackage{mathtools}
\usepackage{listings}
\usepackage{bm}
\newcommand{\uvec}[1]{\boldsymbol{\hat{\textbf{#1}}}}
\usepackage[english]{babel}
\newtheorem{theorem}{Theorem}
\usepackage{setspace}

\setstretch{1.25}
\begin{document}
\section*{3.4 Images and inverse images}
\subsubsection*{Definition 3.4.1 (Images of sets).}
If $f:X\to Y$ is a function from $X$ to $Y$, and $S$ is a set in $X$, we define $f(S)$ to be the set
\begin{equation*}
    f(S):=\{f(x):x\in S\};
\end{equation*}
this set is a subset of $Y$, and is sometimes called the image of $S$ under the map $f$. We sometimes call $f(S)$ the forward image of $S$
to distinguish it from the concept of the inverse image $f^{-1}(S)$ of $S$, which is defined below.

\subsubsection*{Definition 3.4.5 (Inverse images).}
If $U$ is a subset of $Y$, we define the set $f^{-1}(U)$ to be the set 
\begin{equation*}
    f^{-1}(U):=\{x\in X:f(x)\in U\}.
\end{equation*}
In other words, $f^{-1}(U)$ consists of all the elements of $X$ which map into $U$:
\begin{equation*}
    f(x)\in U\iff x\in f^{-1}(U).
\end{equation*}
We feel $f^{-1}(U)$ the inverse image of $U$. 
\subsubsection*{Axiom 3.11 (Power set axiom).}
Let $X$ and $Y$ be sets. Then there exists a set, denoted $Y^X$, which consists of all the functions from $X$ to $Y$, thus
\begin{equation*}
    f\in Y^X\iff (f\text{ is a function with domain }X\text{ and range }Y).
\end{equation*}
\subsubsection*{Lemma 3.4.10}
Let $X$ be a set. Then the set 
\begin{equation*}
    \{Y:Y\text{ is a subset of }X\}
\end{equation*}
is a set.
\subsubsection*{Axiom 3.12 (Union).}
Let $A$ be a set, all of whose elements are themselves sets. Then there exists a set $\bigcup A$ whose elements are precisely those objects
which are elements of the elements of $A$, thus for all objects $x$
\begin{equation*}
    x\in\bigcup A\iff (x\in S\text{ for some }S\in A).
\end{equation*}
\subsection*{Exercises}
\subsubsection*{Exercise 3.4.1}
Let $f:X\to Y$ be a bijective function, and let $f^{-1}:Y\to X$ be its inverse. Let $V$ be any subset of $Y$.
Prove that the forward image of $V$ under $f^{-1}$ is the same set as the inverse image of $V$ under $f$; thus the fact that 
both sets are denoted by $f^{-1}(V)$ will not lead to any inconsistency.
\begin{proof}
    Let $U$ be the forward image of $V$ under $f^{-1}$,
    \begin{equation*}
        U=\{f^{-1}(y):y\in V\}.
    \end{equation*}
    And let $W$ be the inverse image of $V$ under $f$,
    \begin{equation*}
        W=\{x\in X: f(x)\in V\}.
    \end{equation*}
    We need to show that $U=W$ which can be done by proving $x\in U\iff x\in W$. 

    First, consider an arbitrary $x\in U$. Since the range of $f^{-1}$ is $X$, $x\in X$. And there exists exactly one $y\in V$ such that $x=f^{-1}(y)$. 
    By definition of inverse, we have $f(x)=y\in V$. Therefore, $x\in W$.

    Then, consider an arbitrary $x\in W$. Denote $y=f(x)$. Then we have $x\in X$ and $y=f(x)\in Y$. By definition, $x=f^{-1}(y)$. Therefore, $x\in U$.
    
    Thus, $x\in V\iff x\in U$. The statement has been proved. 
\end{proof}
\subsubsection*{Exercise 3.4.2}
Let $f:X\to Y$ be a function from one set $X$ to another set $Y$, let $S$ be a subset of $X$, and let $U$ be a subset of $Y$.
What, in general, can one say about $f^{-1}(f(S))$ and $S$? What about $f(f^{-1}(U))$ and $U$?
\begin{enumerate}
    \item $S\subseteq f^{-1}(f(S))$.
    \begin{proof}
        We need to show that $x\in S\implies x\in f^{-1}(f(S))$. Consider an arbitrary $x\in S$. Then $f(x)\in f(S)$. So $x=f^{-1}(f(x))\in f^{-1}(f(S))$. 
        $f^{-1}(f(S))\subseteq S$ does not stand, see p.58 for a counterexample. Thus, in general, we have $S\subseteq f^{-1}(f(S))$.
    \end{proof}
    \item $f(f^{-1}(U))\subseteq U$.
    \begin{proof}
        We need to show that $y\in f(f^{-1}(U))\implies y\in U$. Consider an arbitrary $y\in f(f^{-1}(U))$. Then there exists $x\in f^{-1}(U)$ such that $f(x)=y$. 
        Since $x\in f^{-1}(U)$, by definition of inverse images, $f(x)=y\in U$. $U\subseteq f(f^{-1}(U))$ is not true, see p.58 for a counterexample. Thus, in general, we have
        $f(f^{-1}(U))\subseteq U$.
    \end{proof}
\end{enumerate}
If $f$ is bijective, we have $S=f^{-1}(f(S))$ and $f(f^{-1}(U))=U$.
\subsubsection*{Exercise 3.4.3}
Let $A,B$ be two subsets of a set $X$, and let $f:X\to Y$ be a function. Show that $f(A\cap B)\subseteq f(A)\cap f(B)$, that $f(A)\backslash f(B)\subseteq f(A\backslash B)$, $f(A\cup B)=f(A)\cup f(B)$. For the first two statements, is it true 
that the $\subseteq$ relation can be improved to $=$?
\begin{enumerate}
    \item $f(A\cap B)\subseteq f(A)\cap f(B)$.
    \begin{proof}
        We need to show that $y\in f(A\cap B)\implies y\in f(A)\cap f(B)$. Assume $y\in f(A\cap B)$, then there exists $x\in A\cap B$ such that $y=f(x)$. $x\in A\cap B\iff (x\in A)\land(x\in B)$. $x\in A\implies y=f(x)\in f(A)$, $x\in B\implies y=f(x)\in f(B)$. 
        So $(y\in f(A))\land(y\in f(B))$. Therefore, $y\in f(A)\cap f(B)$.

        The $\subseteq$ relation cannot be improved to $=$. A counterexample: $A:\{0,1\}$, $B:\{1,2\}$, $f(0)=2$, $f(1)=1$, $f(2)=2$.
    \end{proof}
    \item $f(A)\backslash f(B)\subseteq f(A\backslash B)$.
    \begin{proof}
        We need to show that $y\in f(A)\backslash f(B)\implies y\in f(A\backslash B)$. Assume $y\in f(A)\backslash f(B)$ which means $y\in f(A)\land y\notin f(B)$. Since $y\in f(A)$, there exists $x\in A$ such that $f(x)=y$. 
        On the other hand, $y\notin f(B)$ so $x\notin B$ (otherwise we will have $y=f(x)\in B$). So there exists $(x\in A)\land (x\notin B) \iff x\in (A\backslash B)$ such that $y=f(x)$. Thus, $y\in A\backslash B$.

        The $\subseteq$ relation cannot be improved to $=$. A counterexample: $A:\{1,2\}$, $B:\{2\}$, $f(1)=1$, $f(2)=1$.
    \end{proof}
    \item $f(A\cup B)=f(A)\cup f(B)$.
    \begin{proof}
        We need to show that $y\in f(A\cup B)\iff y\in f(A)\cup f(B)$. 
        
        First, suppose $y\in f(A\cup B)$. Then there exists $x\in A\cup B$ such that $y=f(x)$. $x\in A\cup B\implies (x\in A)\lor (x\in B)$. If $x\in A$, since $y=f(x)$, $y\in f(A)$. If $x\in B$, since $y=f(x)$, $y\in f(B)$.
        So $y\in f(A)$ or $y\in f(B)$. Thus, $y\in f(A)\cup f(B)$. 

        Then, suppose $y\in f(A)\cup f(B)$. If $y\in f(A)$, $\exists x\in A$ such that $y=f(x)$. $x\in A\implies x\in A\cup B$. So $y\in f(A\cup B)$. Similarly, if $y\in f(B)$, we also conclude that $y\in f(A\cup B)$. Therefore, in both cases, we have $y\in f(A\cup B)$.
        Thus, $f(A\cup B)=f(A)\cup f(B)$.
    \end{proof}
\end{enumerate}
\subsubsection*{Exercise 3.4.4}
Let $f:X\to Y$ be a function from one set $X$ to another set $Y$, and let $U,V$ be subsets of $Y$. Show that $f^{-1}(U\cup V)=f^{-1}(U)\cup f^{-1}(V)$, that $f^{-1}(U\cap V)=f^{-1}(U)\cap f^{-1}(V)$, and that $f^{-1}(U\backslash V)=f^{-1}(U)\backslash f^{-1}(V)$.
\begin{enumerate}
    \item $f^{-1}(U\cup V)=f^{-1}(U)\cup f^{-1}(V)$.
    \begin{proof}
        We need to show that $x\in f^{-1}(U\cup V)\iff x\in f^{-1}(U)\cup f^{-1}(V)$. 

        First, suppose $x\in f^{-1}(U\cup V)$. Then there exists $y\in U\cup V$ such that $f(x)=y$. If $y\in U$, $x\in f^{-1}(U)$. If $y\in V$, $x\in f^{-1}(V)$. So $x\in f^{-1}(U)$ or $x\in f^{-1}(V)$. Thus, $x\in f^{-1}(U)\cup f^{-1}(V)$.

        Then, suppose $x\in f^{-1}(U)\cup f^{-1}(V)$ which means $x\in f^{-1}(U)$ or $x\in f^{-1}(V)$. If $x\in f^{-1}(U)$, then $\exists y\in U$ such that $y=f(x)$. If $x\in f^{-1}(V)$, then $\exists y\in V$ such that $y=f(x)$. So $y=f(x)\in U$ or $y=f(x)\in V$. So $y=f(x)\in U\cup V$.
        Thus, $x\in f^{-1}(U\cup V)$. 
        
        Thus, we have shown that $f^{-1}(U\cup V)=f^{-1}(U)\cup f^{-1}(V)$.
    \end{proof}
    \item $f^{-1}(U\cap V)=f^{-1}(U)\cap f^{-1}(V)$.
    \begin{proof}
        We need to show that $x\in f^{-1}(U\cap V)\iff x\in f^{-1}(U)\cap f^{-1}(V)$. 

        First, suppose $x\in f^{-1}(U\cup V)$. Then $\exists y\in U\cap V$ such that $y=f(x)$. Since $y\in U$, $x\in f^{-1}(U)$. Since $y\in V$, $x\in f^{-1}(V)$. And because $x\in f^{-1}(U)$ and $x\in f^{-1}(V)$, $x\in f^{-1}(U)\cap f^{-1}(V)$.

        Then, suppose $x\in f^{-1}(U)\cap f^{-1}(V)$. Then there exists $y=f(x)$ such that $y=f(x)$, $y\in U$ and $y\in V$. So $y=f(x)\in U\cap V$. Thus, $x\in f^{-1}(U\cap V)$.

        Thus, $f^{-1}(U\cap V)=f^{-1}(U)\cap f^{-1}(V)$.
    \end{proof}
    \item $f^{-1}(U\backslash V)=f^{-1}(U)\backslash f^{-1}(V)$.
    \begin{proof}
        We need to show that $x\in f^{-1}(U\backslash V)\iff x\in f^{-1}(U)\backslash f^{-1}(V)$. 
        First, suppose $x\in f^{-1}(U\backslash V)$. Then there exists $(y\in U)\land y\notin V$ such that $f(x)=y$. $y\in U\implies x\in f^{-1}(U)$. On the other hand, $x\notin f^{-1}(V)$ (otherwise $y=f(x)\in V$). 
        So $(x\in f^{-1}(U)) \land (x\notin f^{-1}(V))$. Hence, $x\in f^{-1}(U)\backslash f^{-1}(V)$.

        Then, suppose $x\in f^{-1}(U)\backslash f^{-1}(V)$ which means $x\in f^{-1}(U) \land x\notin f^{-1}(V)$. Since $x\in f^{-1}(U)$, there exists $y\in U$ such that $y=f(x)$. And since $x\notin f^{-1}(V)$, we must have $y\notin V$.
        So there exists $y\in U \land y\notin V \iff y\in U\backslash V$ such that $f(x)=y$. Hence, $x\in f^{-1}(U\backslash V)$.

        Thus, $f^{-1}(U\backslash V)=f^{-1}(U)\backslash f^{-1}(V)$.
    \end{proof}
\end{enumerate}
\subsubsection*{Exercise 3.4.5}
Let $f:X\to Y$ be a function from one set $X$ to another set $Y$. Show that $f(f^{-1}(S))=S$ for every $S\subseteq Y$ if and only if $f$ is surjective. 
Show that $f^{-1}(f(S))=S$ for every $S\subseteq X$ if and only if $f$ is injective.
\begin{enumerate}
    \item $f(f^{-1}(S))=S$ for every $S\subseteq Y$ if and only if $f$ is surjective. 
    \begin{proof}
        We need to show that $y\in f(f^{-1}(S))=S\iff f$ is surjective. And for the LHS, we have proved in 3.4.2 that $f(f^{-1}(S))\subseteq S$ not matter what kind of function $f$ is. So it would be sufficient to show that $S\subseteq f(f^{-1}(S))$. 
        
        First, suppose $f$ is surjective. We want to show that $y\in S\implies y\in f(f^{-1}(S))$. Since $f$ is surjective and $S\in Y$, there must exist $x\in X$ such that $f(x)=y$. Because $y=f(x)$ and $y\in S$, $x\in f^{-1}(S)$. Since $x\in f^{-1}(S)$ 
        and $y=f(x)$, $y\in f(f^{-1}(S))$. 

        Then, suppose $y\in S\implies y\in f(f^{-1}(S))$. We want to show that $f$ is surjective. Assume $f$ is not surjective. Then there exists $y$ and $S\subseteq Y$, such that $y\in S$ and $\forall x\in X$, $f(x)\ne y$. Since $f^{-1}(S)$ is a subset of $X$, 
        for all objects $x\in f^{-1}(S)$, $f(x)\ne y$. Thus, $y\notin f(f^{-1}(S))$, contradiction. Thus, $f$ is surjective.
        
        Thus, $f(f^{-1}(S))=S$ for every $S\subseteq Y$ if and only if $f$ is surjective. 
    \end{proof}
    \item $f^{-1}(f(S))=S$ for every $S\subseteq X$ if and only if $f$ is injective.
    \begin{proof}
        We need to show that $f^{-1}(f(S))=S\iff f$ is injective. For the LHS, it is not necessary to show that $S\subseteq f^{-1}(f(S))$ since we have proved in 3.4.2 that it stands generally. 
        So we only need to show $f^{-1}(f(S))\subseteq S$ for every $S\subseteq X\iff f$ is injective.
        
        First, suppose $f$ is injective. Assume $x\in f^{-1}(f(S))$. Then there exists $y\in f(S)$ such that $y=f(x)$. Since $y\in f(S)$, there exists $x'\in S$ such that $y=f(x')$. 
        And because $f$ is injective, $x=x'$. Therefore, $x\in S$.

        Next, suppose $x\in f^{-1}(f(S))\implies x\in S$. Assume $f$ is not injective. Then $\exists x,x'\in X, x\ne x'$ and $f(x)=f(x')=y$. 
        Let $S$ be $\{x'\}$. In this case, $y\in f(S)$ and $x\in f^{-1}(f(S))$. But $x\notin S$, contradiction. Hence, $f$ is injective.

        Thus, $f^{-1}(f(S))=S$ for every $S\subseteq X$ if and only if $f$ is injective. 
    \end{proof}
\end{enumerate}
\subsubsection*{Exercise 3.4.6}
Prove Lemma 3.4.10. (Hint: start with the set ${\{0,1\}}^X$ and apply the replacement axiom, replacing each function $f$ with the object $f^{-1}(\{1\})$.)
\begin{proof}
    Consider the set ${\{0,1\}}^X$ which is set of all functions that map from $X$ to $\{0,1\}$. Denote ${\{0,1\}}^X$ as $\mathcal{P}(X)$. 
    Let statement $P(f,Y)$ be $Y=f^{-1}(\{1\})$ is a subset of $X$. For any $f$, there exists at most $Y$ for which $P(f, Y)$ is true. 
    Then, by axiom of replacement, there exists a set $\{Y:P(Y)\text{ is true for some }f\in \mathcal{P}(X)\}$, such that for any object $z$,
    \begin{equation*}
        z\in \{Y:P(Y)\text{ is true for some }f\in \mathcal{P}(X)\}\iff P(x,z)\text{ is true for some }f\in\mathcal{P}(X).
    \end{equation*}
    So such a set $Y$ exists and this is exactly the set of all the subsets of $X$. Thus, all the subsets of $X$ is a set.
\end{proof}
\subsubsection*{Exercise 3.4.7}
Let $X,Y$ be sets. Define a partial function from $X$ to $Y$ to be any function $f:X'\to Y'$ whose domain $X'$ is a subset of $X$, and whose range $Y'$ is a subset of $Y$.
Show that the collection of all partial functions from $X$ to $Y$ is itself a set. 
\begin{proof}
    ${\{0,1\}}^X$ is the set of all subsets of $X$, $X'\in {\{0,1\}}^X$. Similarly, $Y'\in {\{0,1\}}^Y$. $Y'^{X'}$ for some $X'\in {\{0,1\}}^X$, $Y'\in {\{0,1\}}^Y$ is the set of 
    all partial functions from $X'$ to $Y'$. Using the union axiom to iterate over all $X'\in{\{0,1\}}^X$ and $Y'\in{\{0,1\}}^Y$ and obtain the set all partial functions.

    Consider an arbitrary $Y_0 \in {\{0,1\}}^Y$. Let ${Y_0}^{X'}$ be the set consists of all the set in the form of ${Y_0}^{X'}$ where $X'\in {\{0,1\}}^X$. (For example, if $X=\{0,1\}$, ${Y_0}^{X'}$ would be $\{{Y_0}^{\{0\}}\},
    {Y_0}^{\{1\}},{Y_0}^{\{0,1\}},\emptyset$.) So every element of ${Y_0}^{X'}$ is a set itself. Apply the union axiom, there exists $\bigcup {Y_0}^{X'}$ whose elements are the elements of the elements of ${Y_0}^{X'}$, 
    \begin{equation*}
        f\in \bigcup {Y_0}^{X'}\iff (f\in S\text{ for some }S\in {Y_0}^{X'}).
    \end{equation*}
    Every $f$ in $\bigcup {Y_0}^{X'}$ is a partial function from $Y_0$ to some $X'$ where $X'\in {\{0,1\}}^X$.

    Then, generalize $Y_0$. Let ${Y'}^{X'}$ be the set of all the sets in the form of $\bigcup {Y'}^{X'}$ where $Y'\in{\{0,1\}}^{Y}$. Every element of ${Y'}^{X'}$ is a set. By the union axiom,
    there exists $\bigcup {Y'}^{X'}$ whose elements are the elements of the elements of ${Y'}^{X'}$, 
    \begin{equation*}
        f\in \bigcup {Y'}^{X'}\iff (f\in S\text{ for some }S\in {Y'}^{X'}).
    \end{equation*}
    Therefore, every $f$ is a partial function from some $Y'\in {\{0,1\}}^Y$ and $X'\in {\{0,1\}}^X$. Thus, $\bigcup {Y'}^{X'}$ is a set, and it is the collection of all partial functions from $X$ to $Y$.
\end{proof}
\subsubsection*{Exercise 3.4.8}
Show that Axiom 3.5 can be deduced from Axiom 3.1, Axiom 3.4 and Axiom 3.12.
\begin{proof}
    Consider the set consists of $A$ and $B$, $\{A, B\}$. This set is a set of sets, by the union axiom, we have 
    \begin{equation*}
        x\in \bigcup\{A,B\}\iff x\in S\text{ for some }S\in\{A,B\}.
    \end{equation*}
    Define $A\cup B$ as $\bigcup\{A,B\}$. Check if $x\in A\cup B\iff x\in A\text{ or }x\in B$ stands.

    Suppose $x\in A\cup B$. Then $x\in S$ for some $S\in \{A,B\}$. By Axiom 3.4, $S=A$ or $S=B$. If $S=A$, $x\in A$. Otherwise, $x\in B$.
    So $x\in A$ or $x\in B$ as desired.

    Suppose $x\in A$ or $x\in B$. Assume $x\in A$. Since $A\in\{A,B\}$, we have $x\in \bigcup\{A,B\}=A\cup B$. Similarly, if $x\in B$ we could also have $x\in \bigcup\{A,B\}=A\cup B$.

    Thus, Axiom 3.5 has been deduced.
\end{proof}
\subsubsection*{Exercise 3.4.9}
Show that if $\beta$ and $\beta'$ are two elements of a set $I$, and to each $\alpha\in I$ we assign a set $A_{\alpha}$, then 
\begin{equation*}
    \{x\in A_{\beta}:x\in A_{\alpha}\text{ for all }\alpha\in I\}=\{x\in A_{\beta'}: x\in A_{\alpha} \text{ for all }\alpha\in I\},
\end{equation*}
and so the definition of $\bigcap_{\alpha\in I}A_{\alpha}$ defined in (3.3) does not depend on $\beta$.
\begin{proof}
    Denote $\{x\in A_{\beta}:x\in A_{\alpha}\text{ for all }\alpha\in I\}$ as $A$ and $\{x\in A_{\beta'}: x\in A_{\alpha} \text{ for all }\alpha\in I\}$ as $A'$. 
    We need to show that $A=A'$ by showing $x\in A\iff x\in A'$. 

    Suppose $x\in A$. Then $x\in A_{\beta}$ and $x\in A_{\alpha}$ for all $\alpha\in I$. Since $\beta'\in I$, $x\in A_{\beta'}$. So $x\in A'$.
    Similarly, we can show that if $x\in A'$, then $x\in A$. Thus, $A=A'$. The definition of $\bigcap_{\alpha\in I}A_{\alpha}$ defined in (3.3) does not depend on $\beta$.
\end{proof} 
\subsubsection*{Exercise 3.4.10}
Suppose that $I$ and $J$ are two sets, and for all $\alpha\in I\cup J$ let $A_{\alpha}$ be a set. Show that $(\bigcup_{\alpha\in I}A_{\alpha})\cup(\bigcup_{\alpha\in J}A_{\alpha})=\bigcup_{\alpha\in I\cup J}A_\alpha$. 
If $I$ and $J$ are non-empty, show that $(\bigcap_{\alpha\in I}A_{\alpha})\cap(\bigcap_{\alpha\in J}A_{\alpha})=\bigcap_{\alpha\in I\cup J}A_{\alpha}$.
\begin{enumerate}
    \item $(\bigcup_{\alpha\in I}A_{\alpha})\cup(\bigcup_{\alpha\in J}A_{\alpha})=\bigcup_{\alpha\in I\cup J}A_\alpha$.
    \begin{proof}
        Show that $x\in(\bigcup_{\alpha\in I}A_{\alpha})\cup(\bigcup_{\alpha\in J}A_{\alpha})\iff x\in\bigcup_{\alpha\in I\cup J}A_\alpha$.

        Suppose $x\in(\bigcup_{\alpha\in I}A_{\alpha})\cup(\bigcup_{\alpha\in J}A_{\alpha})$. Then $x\in \bigcup_{\alpha\in I}A_{\alpha}$ or $x\in\bigcup_{\alpha\in J}A_{\alpha}$. If $x\in \bigcup_{\alpha\in I}A_{\alpha}$, then 
        $x\in A_\alpha$ for some $\alpha_I$. If $x\in\bigcup_{\alpha\in J}A_{\alpha}$, then $x\in A_{\alpha}$ for some $\alpha\in J$. So $x\in A_{\alpha}$ for some $\alpha\in I$ or $\alpha\in J$. In other words, 
        $x\in A_\alpha$ for some $\alpha\in I\cup J$. Therefore, $x\in\bigcup_{\alpha\in I\cup J}A_\alpha$.

        Suppose $x\in\bigcup_{\alpha\in I\cup J}A_\alpha$. Then, $x\in A_\alpha$ for some $\alpha\in I\cup J$, which is equivalent to $x\in A_\alpha$ for some $\alpha\in I$ or some $\alpha\in J$.
        If $x\in A_\alpha$ for some $\alpha\in I$, we have $x\in \bigcup_{\alpha\in I}A_\alpha$. If $x\in A_\alpha$ for some $\alpha\in J$, we have $x\in\bigcup_{\alpha\in J}A_\alpha$. So $x\in \bigcup_{\alpha\in I}A_\alpha$
        or $x\in\bigcup_{\alpha\in J}A_\alpha$. Hence, $x\in(\bigcup_{\alpha\in I}A_{\alpha})\cup(\bigcup_{\alpha\in J}A_{\alpha})$.

        Thus, $(\bigcup_{\alpha\in I}A_{\alpha})\cup(\bigcup_{\alpha\in J}A_{\alpha})=\bigcup_{\alpha\in I\cup J}A_\alpha$.
    \end{proof}
    \item If $I$ and $J$ are non-empty, show that $(\bigcap_{\alpha\in I}A_{\alpha})\cap(\bigcap_{\alpha\in J}A_{\alpha})=\bigcap_{\alpha\in I\cup J}A_{\alpha}$.
    \begin{proof}
        Show that $x\in (\bigcap_{\alpha\in I}A_{\alpha})\cap(\bigcap_{\alpha\in J}A_{\alpha})\iff x\in\bigcap_{\alpha\in I\cup J}A_{\alpha}$.

        Suppose $x\in (\bigcap_{\alpha\in I}A_{\alpha})\cap(\bigcap_{\alpha\in J}A_{\alpha})$. Then we have $x\in A_\alpha$ for all the $\alpha\in I$ and $x\in A_\alpha$ for all the $\alpha\in J$. So $x\in A_\alpha$ for all the $\alpha\in I\cup J$. Therefore, $x\in\bigcap_{\alpha\in I\cup J}A_{\alpha}$. 

        Suppose $x\in \bigcap_{\alpha\in I\cup J}A_{\alpha}$. Then we have $x\in A_\alpha$ for all the $\alpha\in I\cup J$. So there must be $x\in A_\alpha$ for all the $\alpha\in I$ and $x\in A_\alpha$ for all the $\alpha\in J$. Therefore, $x\in(\bigcap_{\alpha\in I}A_{\alpha})\cap(\bigcap_{\alpha\in J}A_{\alpha})$.

        Thus, if $I$ and $J$ are non-empty, show that $(\bigcap_{\alpha\in I}A_{\alpha})\cap(\bigcap_{\alpha\in J}A_{\alpha})=\bigcap_{\alpha\in I\cup J}A_{\alpha}$.
    \end{proof}
\end{enumerate}
\subsubsection*{Exercise 3.4.11}
Let $X$ be a set, let $I$ be a non-empty set, and for all $\alpha\in I$ let $A_\alpha$ be a subset of $X$. Show that 
\begin{equation*}
    X\backslash \bigcup_{\alpha\in I}A_\alpha=\bigcap_{\alpha\in I}(X\backslash A_\alpha)
\end{equation*}
and 
\begin{equation*}
    X\backslash \bigcap_{\alpha\in I}A_\alpha=\bigcup_{\alpha\in I}(X\backslash A_\alpha).
\end{equation*}
This should be compared with de Morgan's laws in Propostition 3.1.27 (although one cannot derive the above identities directly from de Morgan's laws, as $I$ could be infinite).
\begin{enumerate}
    \item $X\backslash \bigcup_{\alpha\in I}A_\alpha=\bigcap_{\alpha\in I}(X\backslash A_\alpha)$.
    \begin{proof}
        We need to show that $x\in X\backslash \bigcup_{\alpha\in I}A_\alpha \iff x\in\bigcup_{\alpha\in I}(X\backslash A_\alpha)$. 

        Suppose $x\in X\backslash \bigcup_{\alpha\in I}A_\alpha$. Then $x\in X$, and $x\notin A_\alpha$ for all the $\alpha\in A$. $x\in X$ is a universal statement so we can rewrite it as
         $x\in X$ for all $\alpha\in A$. So for all $\alpha\in A$, we have $x\in X$ and $x\notin A_\alpha$. Hence, $x\in\bigcap_{\alpha\in I}(X\backslash A_\alpha)$.
    
        Suppose $x\in\bigcap_{\alpha\in I}(X\backslash A_\alpha)$. Then for all $\alpha\in A$ we have $x\in X$ and $x\notin A_\alpha$. Hence, $x\notin \bigcup_{\alpha\in I}A_\alpha$, otherwise there would exist some $\alpha\in I$
        such that $x\in A_\alpha$ (contradiction). Therefore, $x\in X\backslash \bigcup_{\alpha\in I}A_\alpha$.

        Thus, $X\backslash \bigcup_{\alpha\in I}A_\alpha=\bigcap_{\alpha\in I}(X\backslash A_\alpha)$.
        \end{proof}
    \item $X\backslash \bigcap_{\alpha\in I}A_\alpha=\bigcup_{\alpha\in I}(X\backslash A_\alpha)$.
    \begin{proof}
        We need to show that $x\in X\backslash \bigcap_{\alpha\in I}A_\alpha\iff x\in\bigcup_{\alpha\in I}(X\backslash A_\alpha)$.

        Suppose $x\in X\backslash \bigcap_{\alpha\in I}A_\alpha$. Then we have $x\in X$ and $x\notin \bigcap_{\alpha\in I}A_\alpha$. This statement has two parts: for all $\alpha\in A$, $x\in X$, 
        and for some $\alpha\in A$, $x\notin A_\alpha$. We could weaken the first part: for some $\alpha\in A$, $x\in X$. Hence, $x\in X$ and $x\notin A_\alpha$ for some $\alpha\in A$. Therefore,
        $x\in\bigcup_{\alpha\in I}(X\backslash A_\alpha)$.

        Suppose $x\in \bigcup_{\alpha\in I}(X\backslash A_\alpha)$. Then $x\in X$ and $x\notin A_\alpha$ for some $\alpha\in A$. $x\in X$ does not depend on $\alpha$, so if it is true for 
        some $\alpha\in A$, it is true for all $\alpha\in A$. Assume $x\in \bigcap_{\alpha\in I}A_\alpha$. Then $x\in A_\alpha$ for all $\alpha\in A$. On the other hand, $x\notin A_\alpha$ for some $\alpha\in A$, 
        there exists $\alpha\in A$. Contradiction. So $x\notin \bigcap_{\alpha\in I}A_\alpha$. Hence, $x\in X\backslash \bigcap_{\alpha\in I}A_\alpha$.

        Thus, $X\backslash \bigcap_{\alpha\in I}A_\alpha=\bigcup_{\alpha\in I}(X\backslash A_\alpha)$.
    \end{proof}
\end{enumerate}
\end{document} 