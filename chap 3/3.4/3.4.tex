\documentclass[12pt, letter]{article}
\usepackage[utf8]{inputenc}
\usepackage[a4paper, total={6in, 8in}]{geometry}
\usepackage{tikz}
\usepackage[T1]{fontenc}
\usepackage{listings}
\usepackage{graphicx}
\usepackage{amsfonts}
\usepackage{amsmath}
\usepackage{amssymb}
\usepackage{amsthm}
\usepackage{mathtools}
\usepackage{listings}
\usepackage{bm}
\newcommand{\uvec}[1]{\boldsymbol{\hat{\textbf{#1}}}}
\usepackage[english]{babel}
\newtheorem{theorem}{Theorem}
\usepackage{setspace}

\setstretch{1.25}
\begin{document}
\section*{3.4 Images and inverse images}
\subsubsection*{Definition 3.4.1 (Images of sets).}
If $f:X\to Y$ is a function from $X$ to $Y$, and $S$ is a set in $X$, we define $f(S)$ to be the set
\begin{equation*}
    f(S):=\{f(x):x\in S\};
\end{equation*}
this set is a subset of $Y$, and is sometimes called the image of $S$ under the map $f$. We sometimes call $f(S)$ the forward image of $S$
to distinguish it from the concept of the inverse image $f^{-1}(S)$ of $S$, which is defined below.

\subsubsection*{Definition 3.4.5 (Inverse images).}
If $U$ is a subset of $Y$, we define the set $f^{-1}(U)$ to be the set 
\begin{equation*}
    f^{-1}(U):=\{x\in X:f(x)\in U\}.
\end{equation*}
In other words, $f^{-1}(U)$ consists of all the elements of $X$ which map into $U$:
\begin{equation*}
    f(x)\in U\iff x\in f^{-1}(U).
\end{equation*}
We feel $f^{-1}(U)$ the inverse image of $U$. 
\subsubsection*{Axiom 3.11 (Power set axiom).}
Let $X$ and $Y$ be sets. Then there exists a set, denoted $Y^X$, which consists of all the functions from $X$ to $Y$, thus
\begin{equation*}
    f\in Y^X\iff (f\text{ is a function with domain }X\text{ and range }Y).
\end{equation*}
\subsubsection*{Lemma 3.4.10}
Let $X$ be a set. Then the set 
\begin{equation*}
    \{Y:Y\text{ is a subset of }X\}
\end{equation*}
is a set.
\subsubsection*{Axiom 3.12 (Union).}
Let $A$ be a set, all of whose elements are themselves sets. Then there exists a set $\bigcup A$ whose elements are precisely those objects
which are elements of the elements of $A$, thus for all objects $x$
\begin{equation*}
    x\in\bigcup A\iff (x\in S\text{ for some }S\in A).
\end{equation*}
\subsection*{Exercises}
\subsubsection*{Exercise 3.4.1}
Let $f:X\to Y$ be a bijective function, and let $f^{-1}:Y\to X$ be its inverse. Let $V$ be any subset of $Y$.
Prove that the forward image of $V$ under $f^{-1}$ is the same set as the inverse image of $V$ under $f$; thus the fact that 
both sets are denoted by $f^{-1}(V)$ will not lead to any inconsistency.
\begin{proof}
    Let $U$ be the forward image of $V$ under $f^{-1}$,
    \begin{equation*}
        U=\{f^{-1}(y):y\in V\}.
    \end{equation*}
    And let $W$ be the inverse image of $V$ under $f$,
    \begin{equation*}
        W=\{x\in X: f(x)\in V\}.
    \end{equation*}
    We need to show that $U=W$ which can be done by proving $x\in U\iff x\in W$. 

    First, consider an arbitrary $x\in U$. Since the range of $f^{-1}$ is $X$, $x\in X$. And there exists exactly one $y\in V$ such that $x=f^{-1}(y)$. 
    By definition of inverse, we have $f(x)=y\in V$. Therefore, $x\in W$.

    Then, consider an arbitrary $x\in W$. Denote $y=f(x)$. Then we have $x\in X$ and $y=f(x)\in Y$. By definition, $x=f^{-1}(y)$. Therefore, $x\in U$.
    
    Thus, $x\in V\iff x\in U$. The statement has been proved. 
\end{proof}
\subsubsection*{Exercise 3.4.2}
Let $f:X\to Y$ be a function from one set $X$ to another set $Y$, let $S$ be a subset of $X$, and let $U$ be a subset of $Y$.
What, in general, can one say about $f^{-1}(f(S))$ and $S$? What about $f(f^{-1}(U))$ and $U$?
\begin{enumerate}
    \item $S\subseteq f^{-1}(f(S))$.
    \begin{proof}
        We need to show that $x\in S\implies x\in f^{-1}(f(S))$. Consider an arbitrary $x\in S$. Then $f(x)\in f(S)$. So $x=f^{-1}(f(x))\in f^{-1}(f(S))$. 
        $f^{-1}(f(S))\subseteq S$ does not stand, see p.58 for a counterexample. Thus, in general, we have $S\subseteq f^{-1}(f(S))$.
    \end{proof}
    \item $f(f^{-1}(U))\subseteq U$.
    \begin{proof}
        We need to show that $y\in f(f^{-1}(U))\implies y\in U$. Consider an arbitrary $y\in f(f^{-1}(U))$. Then there exists $x\in f^{-1}(U)$ such that $f(x)=y$. 
        Since $x\in f^{-1}(U)$, by definition of inverse images, $f(x)=y\in U$. $U\subseteq f(f^{-1}(U))$ is not true, see p.58 for a counterexample. Thus, in general, we have
        $f(f^{-1}(U))\subseteq U$.
    \end{proof}
\end{enumerate}
If $f$ is bijective, we have $S=f^{-1}(f(S))$ and $f(f^{-1}(U))=U$.
\end{document}