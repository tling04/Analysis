\documentclass[12pt, letter]{article}
\usepackage[utf8]{inputenc}
\usepackage[a4paper, total={6in, 8in}]{geometry}
\usepackage{tikz}
\usepackage[T1]{fontenc}
\usepackage{listings}
\usepackage{graphicx}
\usepackage{amsfonts}
\usepackage{amsmath}
\usepackage{amssymb}
\usepackage{amsthm}
\usepackage{mathtools}
\usepackage{listings}
\usepackage{bm}
\newcommand{\uvec}[1]{\boldsymbol{\hat{\textbf{#1}}}}
\usepackage[english]{babel}
\newtheorem{theorem}{Theorem}
\usepackage{setspace}

\setstretch{1.25}
\begin{document}
\section*{3.3 Functions}
\subsubsection*{Definition 3.3.1 (Functions).}
Let $X,Y$ be sets, and let $P(x,y)$ be a property pertaining to an object $x\in X$ and an object $y\in Y$, such that
for every $x\in X$, there is exactly one $y\in Y$ for which $P(x,y)$ is true (this is sometimes known as the vertical line test).
Then we define the function $f: X\rightarrow Y$ defined by $P$ on the domain $X$ and range $Y$ to be the object which, 
given any input $x\in X$, assigns an output $f(x)\in Y$, defined to be the unique object $f(x)$ for which $P(x,f(x))$ is true. 
Thus, for any $x\in X$ and $y\in Y$,
\begin{equation*}
    y=f(x)\iff P(x,y) \text{ is true.}
\end{equation*}

\subsubsection*{Definition 3.3.7 (Equality of functions).}
Two functions $f:X\rightarrow Y$, $g:X\rightarrow Y$ with the same domain and range are said to be equal, $f=g$,
if and only if $f(x)=g(x)$ for all $x\in X$. If $f(x)$ and $g(x)$ agree for some values of $x$, but not others, then we do not consider $f$ and 
$g$ to be equal. If two functions $f,g$ have different domains, or different ranges, we also do not consider them to be equal.

\subsubsection*{Definition 3.3.11 (Composition).}
Let $f:X\to Y$ and $g:Y\to Z$ be two functions, such that the range of $f$ is the same set as the domain of $g$. We then define the composition
$g\circ f:X\to Z$ of the two functions $g$ and $f$ to be the function defined explicitly by the formula
\begin{equation*}
    (g\circ f)(x):=g(f(x)).
\end{equation*}
If the range of $f$ does not match the domain of $g$, we leave the composition $g\circ f$ undefined.
\subsubsection*{Lemma 3.3.13 (Composition is associative).}
Let $f:Z\to W$, $g:Y\to Z$, and $h:X\to Y$ be functions. Then $f\circ(g\circ h)=(f\circ g)\circ h$.
\subsubsection*{Definition 3.3.15 (One-to-one functions).}
A function $f$ is one-to-one (or injective) if different elements map to different elements:
\begin{equation*}
    x\ne x'\implies f(x)\ne f(x').
\end{equation*}
Equivalently, a function is one-to-one if 
\begin{equation*}
    f(x)=f(x')\implies x=x'.
\end{equation*}
\subsubsection*{Definition 3.3.18 (Onto functions).}
A function $f$ is onto (or surjective) if every element if $Y$ comes from applying $f$ to some element in $X$:
\begin{equation*}
    \text{For every }y\in Y,\text{ there exists }x\in X \text{ such that }f(x)=y.
\end{equation*}
\subsubsection*{Definition 3.3.21 (Bijective functions).}
Functions $f:X\to Y$ which are both one-to-one and onto are also called bijective or invertible.


\subsection*{Exercises}
\subsubsection*{Exercise 3.3.1}
Show that the definition of equality in Definition 3.3.7 is reflexive, symmetric, and transitive. Also verify the substitution property: if 
$f,\tilde{f}:X\to Y$ and $g,\tilde{g}:Y\to Z$ are functions such that $f=\tilde{f}$ and $g=\tilde{g}$, then $g\circ f=\tilde{g}\circ\tilde{f}$.
\begin{proof}
    Reflexivity: $f$ and $f$ have the same domain and range, and $f(x)=f(x)$ for all $x$ in the domain of $f$. Therefore, $f$ is equal to itself.

    Symmetry: $g$ and $f$ have the same domain and range. For every $x$ in the domain of $g$, we have $g(x)=f(x)$. Therefore, by Definition 3.3.7, $g(x)$ and 
    $f(x)$ are equal.

    Transitivity: Suppose $f$ and $g$ have the same domain and range, and for every $x$ in the domain of $f$, $f(x)=g(x)$. And $g$ and $h$ have the same domain and range, and 
    for every $x$ in the domain of $g$, we have $g(x)=h(x)$. Then $f$ and $h$ have the same domain and range. $\forall x$ in the domain of $f$, we have $f(x)=g(x)=h(x)$. Therefore, 
    $f$ and $h$ are equal.

    Substitution property: Since $g\circ f, \tilde{g}\circ \tilde{f}:X\to Z$, they have the same domain and range. And for every $x\in X$, we have $f(x)=\tilde{f}(x)$, since $g=\tilde{g}$, we also have
    $g(f(x))=\tilde{g}(f(x))=\tilde{g}(\tilde{f}(x))$. Therefore, $g\circ f=\tilde{g}\circ\tilde{f}$.
\end{proof}
\subsubsection*{Exercise 3.3.2}
Let $f:X\to Y$ and $g:Y\to Z$ be functions. Show that if $f$ and $g$ are both injective, then so is $g\circ f$; similarly, show that if $f$ and $g$ are both 
surjective, then so is $g\circ f$.
\begin{enumerate}
    \item If if $f$ and $g$ are both injective, then so is $g\circ f$.
    \begin{proof}
        $f$ is injective: 
        \begin{equation*}
            x\in X, x'\in X, x\ne x'\implies f(x)\ne f(x').
        \end{equation*}
        $g$ is injective:
        \begin{equation*}
            f(x)\in Y, f(x')\in Y, f(x)\ne f(x')\implies g(f(x))\ne g(f(x')).
        \end{equation*}
        Therefore, $x\ne x'\implies (g\circ f)(x)\ne (g\circ f)(x')$. Thus, $g\circ f$ is injective.
    \end{proof}
    \item If $f$ and $g$ are both surjective, then so is $g\circ f$.
    \begin{proof}
        $f$ is surjective:
        \begin{equation*}
            \text{For every }y\in Y, \text{ there exists }x\in X\text{ such that }f(x)=y.
        \end{equation*}
        $g$ is surjective:
        \begin{equation*}
            \text{For every }z\in Z, \text{ there exists }y\in Y\text{ such that }g(y)=z.
        \end{equation*}
        Therefore, for every $z\in Z$, there exists $x\in X$ such that $(g\circ f)(x)=g(f(x))=g(y)=z$. Thus, $g\circ f$ is surjective.
    \end{proof}
\end{enumerate}
\subsubsection*{Exercise 3.3.3}
When is the empty function injective? surjective? bijective?

The empty function is of the form $f:\emptyset\to X$. It is always injective no matter what $X$ is. It is surjective if $X$ is $\emptyset$. It is 
bijective if $X$ is $\emptyset$.
\subsubsection*{Exercise 3.3.4}
In this section we give some cancellation laws for composition. Let $f:X\to Y$, $\tilde{f}:X\to Y$, $g:Y\to Z$, and $\tilde{g}:Y\to Z$ be functions.
Show that if $g\circ f=g\circ \tilde{f}$ and $g$ is injective, then $f=\tilde{f}$. Is the same statement true if $g$ is not injective?
Show that if $g\circ f=\tilde{g}\circ f$ and $f$ is surjective, then $g=\tilde{g}$. Is the same statement true if $f$ is not surjective?
\begin{enumerate}
    \item 
    \begin{proof}
        Suppose $x$ is an arbitrary object in $X$, $y=f(x)\in Y$, $y'=\tilde{f}(x)\in Y$. Since $g\circ f=g\circ\tilde{f}$, $g(f(x))=g(\tilde{f}(x))$. Because $g$ is injective, 
    $g(f(x))=g(\tilde{f}(x))\implies f(x)=\tilde{f}(x)$. And $f,\tilde{f}$ have the same domain and range. Thus, $f=\tilde{f}$.

    This won't be true if $g$ is not injective. Counterexample: $g(1)=3, g(2)=3, f(1)=1, \tilde{f}(1)=2$. In this case, $g(f(1))=g(\tilde{f}(1))=3$, but $f\ne\tilde{f}$.
    \end{proof}
    \item 
    \begin{proof}
        Since $f$ is surjective, for every $y\in Y$, there exists $x\in X$ such that $y=f(x)$. Also for every $x\in X$, we have $g(f(x))=\tilde{g}(f(x))$ as $g\circ f=g\circ\tilde{f}$.
    Therefore, for every $y\in Y$, there exists $x\in X$ such that $g(y)=g(f(x))=\tilde{g}(f(x))=\tilde{g}(y)$. As $g,\tilde{g}$ have the same domain and range,
    $g=\tilde{g}$.

    This won't be true if $f$ is not surjective. Counterexample: $f:\{0,1\}\to\{1,2,3\}$, $g:\{1,2,3\}\to\{4,5,6\}$, $\tilde{g}:\{1,2,3\}\to\{4,5,7\}$.
    \end{proof}
\end{enumerate}
\subsubsection*{Exercise 3.3.5}
Let $f:X\to Y$ and $g:Y\to Z$ be functions. Show that if $g\circ f$ is injective, then $f$ must be injective. Is it true that $g$ must also be injective?
Show that if $g\circ f$ is surjective, then $g$ must be surjective. Is it true that $f$ must also be surjective?
\begin{enumerate}
    \item 
    \begin{proof}
    $g\circ f$ is injective $\implies (x\ne x'\implies g(f(x))\ne g(f(x')))$. Suppose $f$ is not injective, that is, $\exists x,x'\in X$ such that $x\ne x'$ and $f(x)=f(x')$. Then by definition, $g(f(x))=g(f(x'))$ so  
    $g\circ f$ is not injective. (contradiction) Therefore, $f$ must be injective.  However, $g$ does not have to be injective. Counterexample: $f:\{0,1\}\to\{1,2,3\}$, $g:\{1,2,3\}\to\{4,5,5\}$. ($f$ does not have to be surjective.)
    \end{proof}    
    \item
    \begin{proof}
        $g\circ f$ is surjective $\implies \forall z\in Z,\exists x\in X$ such that $g(f(x))=z$. 
        Assume $g$ is not surjective. Then $\exists z\in Z$ such that $\forall y\in Y, g(y)\ne z$. So there does not exist $x\in X$ such that $g(y)=g(f(x))=z$. It implies $g\circ f$ is not surjective. (contradiction) Thus, $g$ must be surjective.
        $f$ does not have to be surjective. Counterexample: $f:\{1\}\to\{2,3\},f(1):=2,g:\{2,3\}\to\{4\},g(2):=4,g(3):=4$. ($g$ does not have to be injective.)
    \end{proof}
\end{enumerate}
\subsubsection*{Exercise 3.3.6}
Let $f:X\to Y$ be a bijective function, and let $f^{-1}:Y\to X$ be its inverse. Verify the cancellation laws $f^{-1}(f(x))=x$ for all $x\in X$ and $f(f^{-1}(y))=y$ for all $y\in Y$. Conclude that $f^{-1}$ is also invertible, and has $f$ as its inverse (thus $(f^{-1})^{-1}=f$).
\begin{proof}
    Since $f$ is surjective, for every $y\in Y$, there exists $x\in X$ such that $f(x)=y$. Suppose $\exists x,x'\in X$ such that $f^{-1}(f(x))=x'\ne x$. Then there exists $y\in Y$, such that $f(x)=y$ and $f^{-1}(y)=x' (f(x')=y)$. So $f$ is not injective. (contradiction) Thus, $f^{-1}(f(x))=x$ for all $x\in X$.
    
    Since $f$ is surjective, for every $y\in Y$, there exists $x\in X$ such that $f^{-1}(y)=x$. Suppose $f(x)=y$ and $\exists y'\in Y,y'\ne y, f(f^{-1}(y))=y'$. Then we have $f(f^{-1}(y))=f(x)=y'\ne y$ (contradiction). Thus, $f(f^{-1}(y))=y$. 

    Since $f^{-1}(f(x))=x$ for all $x\in X$, $f^{-1}$ is surjective. Suppose there exists $y,y'\in Y$, $x\in X$, $y\ne y'$, $f^{-1}(y)=f^{-1}(y')=x$. By cancellation law, we have $f(x)=y$ and $f(x)=y'$ (contradiction). Therefore, $f^{-1}$ is injective. 
    Thus, $f^{-1}$ is bijective. 

    $(f^{-1})^{-1}$ has $X$ as its domain and $Y$ as its range, the inverse of $f^{-1}$. Then we need to show that for every $x\in X$, we have $f(x)=(f^{-1})^{-1}(x)$. Let $x$ be an arbitrary object in $X$, $y=f(x)$. By definition of inverse, $f^{-1}(y)=x$. 
    Again by definition of inverse, $(f^{-1})^{-1}(x)=y$. Therefore, $f(x)=(f^{-1})^{-1}(x)$. Thus, $(f^{-1})^{-1}=f$.
\end{proof}
\subsubsection*{Exercise 3.3.7}
Let $f:X\to Y$ and $g:Y\to Z$ be functions. Show that if $f$ and $g$ are bijective, then so is $g\circ f$, and we have $(g\circ f)^{-1}=f^{-1}\circ g^{-1}$.
\begin{proof}
    We have shown in 3.3.2, if $f$ and $g$ are both injective/surjective, $f\circ g$ is also injective/surjective. By symmetry, $g\circ f$ is also injective/surjective. As bijective $\iff$ injective and surjective, $f$ and $g$ are bijective $\iff$ $g\circ f$ is bijective. 

    $g\circ f:X\to Z$, so $(g\circ f)^{-1}:Z\to X$. Since $f:X\to Y$ and $g:Y\to Z$, $g^{-1}:Z\to Y$, $f:Y\to X$, we have $f^{-1}\circ g^{-1}:Z\to X$. Therefore, $(g\circ f)^{-1}$ and $f^{-1}\circ g^{-1}$ have the same domain and range. Then consider an arbitrary obejct $z\in Z$. Since $g$ and $f$ are both bijective, there exist exactly one $x$ and one $y$ such that 
    $g(y)=z$ and $f(x)=y$. So $(g\circ f)(x)=z$. And by definition of inverse, we have $(g\circ f)^{-1}(z)=x$. Again by definition, we have $g^{-1}(z)=y$, $f^{-1}(y)=x$. So $(f^{-1}\circ g^{-1})(z)=x$. Therefore, for every $z\in Z$, $(g\circ f)^{-1}(z)=(f^{-1}\circ g^{-1})(z)$. Thus, $(g\circ f)^{-1}=f^{-1}\circ g^{-1}$.
\end{proof}
\subsubsection*{Exercise 3.3.8}
If $X$ is a subset of $Y$, let $\iota_{X\to Y}$ be the inclusion map from $X$ to $Y$, defined by mapping $x\mapsto x$ for all $x\in X$, i.e., $\iota_{X\to Y}(x):=x$ for all $x\in X$.
The map $\iota_{X\to X}$ is in particular called the identity map on $X$.
\begin{enumerate}
    \item Show that if $X\subseteq Y\subseteq Z$ then $\iota_{Y\to Z}\circ\iota_{X\to Y}=\iota_{X\to Z}$.
    \begin{proof}
        Both $\iota_{Y\to Z}\circ\iota_{X\to Y}$ and $\iota_{X\to Z}$ have $X$ as domain and $Z$ as range. Consider an arbitrary object $x\in X$. $\iota_{X\to Z}(x)=x$. 
        $\iota_{X\to Y}(x)=x$, $(\iota_{Y\to Z}\circ\iota_{X\to Y})(x)=\iota_{Y\to Z}(\iota_{X\to Y}(x))=\iota_{Y\to Z}(x)=x$. Therefore, for every $x\in X$, $(\iota_{Y\to Z}\circ\iota_{X\to Y})(x)=\iota_{X\to Z}(x)$.
        Thus, $\iota_{Y\to Z}\circ\iota_{X\to Y}=\iota_{X\to Z}$.
    \end{proof}
    \item Show that if $f:A\to B$ is any function, then $f=f\circ\iota_{A\to A}=\iota_{B\to B}\circ f$.
    \begin{proof}
        Obviously, $f$, $f\circ\iota_{A\to A}$, and $\iota_{B\to B}\circ f$ all have the same domain and range. Consider an arbitrary $x\in A$. $(f\circ\iota_{A\to A})(x)=f(\iota_{A\to A}(x))=f(x)$. 
        $(\iota_{B\to B}\circ f)(x)=\iota_{B\to B}(f(x))=f(x)$. Therefore, for every $x\in A$, $f(x)=(f\circ\iota_{A\to A})(x)=(\iota_{B\to B}\circ f)(x)$. Thus, $f=f\circ\iota_{A\to A}=\iota_{B\to B}\circ f$.
    \end{proof}
    \item Show that, if $f:A\to B$ is a bijective function, then $f\circ f^{-1}=\iota_{B\to B}$ and $f^{-1}\circ f=\iota_{A\to A}$.
    \begin{enumerate}
        \item $f\circ f^{-1}=\iota_{B\to B}$.
        \begin{proof}
            $f^{-1}:B\to A$, $f\circ f^{-1}:B\to B$. So $f\circ f^{-1}$ and $\iota_{B\to B}$ have the same domain and range. Consider an arbitrary $y\in B$. $\iota_{B\to B}(y)=y$. Since $f$ is bijective, there exists exactly one $x\in A$ such that $f(x)=y$. 
            Then $(f\circ f^{-1})(y)=f(f^{-1}(y))=f(x)=y=\iota_{B\to B}(y)$. Thus, $f\circ f^{-1}=\iota_{B\to B}$.
        \end{proof}
        \item $f^{-1}\circ f=\iota_{A\to A}$.
        
        \begin{proof}
            $f:A\to B$, $f^{-1}\circ f: B\to A$. So $f^{-1}\circ f$ and $\iota_{A\to A}$ have the same domain and range. Consider an arbitrary $x\in A$. $\iota_{A\to A}(x)=x$. Let $f(x)=y$. By definition of inverse, $f^{-1}(y)=x$. So $(f^{-1}\circ f)(x)=f^{-1}(f(x))=f^{-1}(y)=x=\iota_{A\to A}(x)$. Thus, 
            $f^{-1}\circ f=\iota_{A\to A}$.
        \end{proof}
    \end{enumerate}
    \item Show that if $X$ and $Y$ are disjoint sets, and $f:X\to Z$ and $g:Y\to Z$ are functions, then there is a unique function $h:X\cup Y\to Z$ such that 
    $h\circ\iota_{X\to X\cup Y}=f$ and $h\circ\iota_{Y\to X\cup Y}=g$.
    \begin{proof}
        The existence of $h$: $\forall x\in X\cup Y$, if $x\in X$, $h(x):=f(x)$, if $x\in Y$, $h(x):=g(x)$. 
        Then we can know that $h\circ\iota_{X\to X\cup Y}$ and $f$ both have domain $X$ and range $Z$. For an arbitrary $x\in X$, 
        $(h\circ\iota_{X\to X\cup Y})(x)=h(\iota_{X\to X\cup Y}(x))=f(x)$. Thus, $h\circ\iota_{X\to X\cup Y}=f$. Similarly, we can show that 
        $h\circ\iota_{Y\to X\cup Y}=g$. To check the uniqueness of $h$, we need to show that if there exists another function $h'$ with the same domain, range and properties as $h$, then $h'=h$. 
        Consider an arbitrary $x\in X$. $(h'\circ\iota_{X\to X\cup Y})(x)=h'(\iota_{X\to X\cup Y}(x))=h'(x)$. Since $h'\circ\iota_{X\to X\cup Y}=f$, we must have $(h'\circ\iota_{X\to X\cup Y})(x)=f(x)=(h\circ\iota_{X\to X\cup Y})(x)$. 
        Similarly, we can show that for every $y\in Y$, we have $(h'\circ\iota_{Y\to X\cup Y})(y)=g(y)=(h\circ\iota_{Y\to X\cup Y})(y)$. Since $h'$ and $h$ have the same domain and range, we can conclude that $h=h'$. Thus, $h$ is unique.
    \end{proof}
\end{enumerate}
\end{document}