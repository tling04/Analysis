\documentclass[12pt, letter]{article}
\usepackage[utf8]{inputenc}
\usepackage[a4paper, total={6in, 8in}]{geometry}
\usepackage{tikz}
\usepackage[T1]{fontenc}
\usepackage{listings}
\usepackage{graphicx}
\usepackage{amsfonts}
\usepackage{amsmath}
\usepackage{amssymb}
\usepackage{amsthm}
\usepackage{mathtools}
\usepackage{listings}
\usepackage{bm}
\newcommand{\uvec}[1]{\boldsymbol{\hat{\textbf{#1}}}}
\usepackage[english]{babel}
\newtheorem{theorem}{Theorem}
\usepackage{setspace}

\setstretch{1.25}
\begin{document}
\section*{3.6 Cardinality of sets}
\subsubsection*{Definition 3.6.1 (Equal cardinality).}
We say that two sets $X$ and $Y$ have equal cardinality iff there exists a bijection $f:X\to Y$ from $X$ to $Y$.
\subsubsection*{Proposition 3.6.4}
Let $X,Y,Z$ be sets. Then $X$ has equal cardinality with $X$. If $X$ has equal cardinality with $Y$, then $Y$ has equal cardinality with $X$.
If $X$ has equal cardinality with $Y$ and $Y$ has equal cardinality with $Z$, then $X$ has equal cardinality with $Z$.
\subsubsection*{Definition 3.6.5}
Let $n$ be a natural number. A set $X$ is said to have cardinality $n$, iff it has equal cardinality with $\{i\in\mathbf{N}:1\leq i\leq n\}$. We also say that $X$ has
$n$ elements iff it has cardinality $n$.
\subsubsection*{Proposition 3.6.8 (Uniqueness of cardinality).}
Let $X$ be aset with some cardinality $n$. Then $X$ cannot have any other cardinality, i.e., $X$ cannot have cardinality $m$ for any $m\ne n$. 
\subsubsection*{Lemma 3.6.9}
Suppose that $n\geq 1$, and $X$ has cardinality $n$. Then $X$ is non-empty, and if $x$ is any element of $X$, then the set $X-\{x\}$ (i.e., $X$ with the element $x$ removed)
has cardinality $n-1$.
\subsubsection*{Definition 3.6.10 (Finite sets).}
A set is finite iff it has cardinality $n$ for some natural number $n$; otherwise, the set is called infinite. If $X$ is a finite set, we use $\#(X)$
to denote the cardinality of $X$.
\subsubsection*{Theorem 3.6.12}
The set of natural numbers $\mathbf{N}$ is infinite.
\subsubsection*{Proposition 3.6.14 (Cardinal arithmetic).}
See Exercise 3.6.4.
\subsection*{Exercises}
\subsubsection*{Exercise 3.6.1}
Prove Proposition 3.6.4.
\begin{itemize}
    \item $X$ has equal cardinality with $X$.
    \begin{proof}
        Define function $f:X\to X$ such that for each $x\in X$, $f(x)=x$. For $x_1\ne x_2$, $f(x_1)=x_1$ and $f(x_2)=x_2$. So $f(x_1)\ne f(x_2)$. Therefore,
        $f$ is injective. By definition, for every $x\in X$, $f(x)=x$. So $f$ is surjective. Thus, $f$ is bijective and $X$ has equal cardinality with $X$.
    \end{proof}
    \item If $X$ has equal cardinality with $Y$, then $Y$ has equal cardinality with $X$.
    \begin{proof}
        Since $X$ has equal cardinality with $Y$, there exists a bijective function $f:X\to Y$. Since $f$ is bijective, there exists $f^{-1}:Y\to X$. For $y_1,y_2\in Y$, 
        if we have $f^{-1}(y_1)=f^{-1}(y_2)$, by the definition of function, $f(f^{-1}(y_1))=f(f^{-1}(y_2))$, then $y_1=y_2$. So $f^{-1}$ is injective.
        For every $x\in X$, we have $f(x)\in Y$ such that $f^{-1}(f(x))=x$. So $f^{-1}$ is surjective. Thus, $f^{-1}$ is bijective and $Y$ has equal cardinality with $X$.
    \end{proof}
    \item If $X$ has equal cardinality with $Y$ and $Y$ has equal cardinality with $Z$, then $X$ has equal cardinality with $Z$.
    \begin{proof}
        Since $X$ has equal cardinality with $Y$, there exists a bijective function $f:X\to Y$. Since $Y$ has equal cardinality with $Z$, there exists a bijective function $g:Y\to Z$.
        By Exercise 3.3.7, $g\circ f:X\to Z$ is also bijective. Thus, $X$ has equal cardinality with $Z$.
    \end{proof}
\end{itemize}
\subsubsection*{Exercise 3.6.2}
Show that a set $X$ has cardinality 0 if and only if $X$ is the empty set.
\begin{proof}
    By definition 3.6.5, $X$ has $n$ elements iff it has cardinality $n$. So since $X$ has cardinality 0, it has no element in it which means $X$ is the empty set.
    On the other hand, if $X$ is the empty set, it has 0 element and thus has cardinality 0.
\end{proof}
\subsubsection*{Exercise 3.6.3}
Let $n$ be a natural number, and let $f:\{i\in\mathbf{N}:1\leq i\leq n\}\to\mathbf{N}$ be a function. Show that there exists a natural number $M$ such that $f(i)\leq M$ for all $1\leq i\leq n$.
Thus finite subsets of the natural numbers are bounded.
\begin{proof}
   For function $f:\{i\in\mathbf{N}:1\leq i\leq n\}\to \mathbf{N}$, we claim that $M=\max\{f(1),\dotsc,f(n)\}$. Induct on $n$. Base case: $n=1$. Let $M=\max\{f(1)\}=f(1)$. $M\leq f(i)$ for $1\leq i\leq 1$. This proves the base case. Then suppose inductively that there exists $M'=
    \max\{f(1),\dotsc, f(n)\}$ is the upper bound for $f:\{i\in\mathbf{N}:1\leq i\leq n\}\to\mathbf{N}$. Now consider $f:\{i\in\mathbf{N}:1\leq i\leq n+1\}\to\mathbf{N}$. Let $M=\max\{M',f(n+1)\}$. 
    For all $1\leq i\leq n$, we have $f(i)\leq M'\leq M$. Also we have $f(n+1)\leq M$. Thus, $f(i)\leq M$ for all $1\leq i\leq n+1$. This closes the induction.
\end{proof}
\subsubsection*{Exercise 3.6.4}
Prove proposition 3.6.14.
\begin{enumerate}
    \item Let $X$ be a finite set, and let $x$ be an object which is not an element of $X$. Then $X\cup\{x\}$ is finite and $\#(X\cup\{x\})=\#(X)+1$.
    \begin{proof}
        Use $n$ to denote the cardinality of $X$. By Lemma 3.6.9, $\#(X)=\#((X\cup\{x\})-\{x\})=\#(X\cup\{x\})-1$. So $\#(X\cup\{x\})=\#(X)+1=n+1$ which is also a natural number.
        Thus, $X\cup\{x\}$ is finite and $\#(X\cup\{x\})=\#(X)+1$.
    \end{proof}
    \item Let $X$ and $Y$ be finite sets. Then $X\cup Y$ is finite and $\#(X\cup Y)\leq \#(X)+\#(Y)$. If in addition $X$ and $Y$ are disjoint, then $\#(X\cup Y)=\#(X)+\#(Y)$.
    \begin{proof}
        Use $m$ to denote the cardinality of $X=\{x_1,\dotsc, x_m\}$ and $n$ to denote the cardinality of $Y=\{y_1\dotsc, y_n\}$. Induct on $n$. The base case is when $\#(Y)=n=0$. $\#(X\cup Y)=\#(X)\leq \#(X)+\#(Y)=\#(X)$.
        Now suppose inductively $\#(X\cup Y)\leq \#(X)+\#(Y)$ when $\#(Y)=n$ ($Y=\{y_1,\dotsc, y_n\}$). Consider when $Y=\{y_1,\dotsc,y_n,y_{n+1}\}$ and $\#(Y)=n+1$. 
        If $y_{n+1}\in \{x_1,\dotsc, x_m\}\cup\{y_1,\dotsc, y_n\}$, then 
        \begin{equation*}
            \begin{gathered}
                \#(\{x_1,\dotsc,x_m\}\cup\{y_1,\dotsc,y_{n+1}\})=\#(\{x_1,\dotsc,x_m\}\cup\{y_1,\dotsc,y_{n}\})\\ \leq m+n<m+(n+1)=\#(X)+\#(Y).
            \end{gathered}
        \end{equation*}
        So in this case, $\#(X\cup Y)<\#(X)+\#(Y)$. If $y_{n+1}\notin \{x_1,\dotsc, x_m\}\cup\{y_1,\dotsc, y_n\}$, then 
        \begin{equation*}
            \begin{gathered}
                \#(\{x_1,\dotsc,x_m\}\cup\{y_1,\dotsc,y_{n+1}\})=\#(\{x_1,\dotsc,x_m\}\cup\{y_1,\dotsc,y_{n}\})+1,
            \end{gathered}
        \end{equation*}
        since by induction hypothesis we have,
        \begin{equation*}
            \#(\{x_1,\dotsc,x_m\}\cup\{y_1,\dotsc,y_{n}\})\leq m+n,
        \end{equation*}
        then 
        \begin{equation*}
            \#(\{x_1,\dotsc,x_m\}\cup\{y_1,\dotsc,y_{n+1}\})\leq m+(n+1)=\#(X)+\#(Y).
        \end{equation*}
        So in this case, $\#(X\cup Y)\leq \#(X)+\#(Y)$. Thus, in both cases, we have $\#(X\cup Y)\leq \#(X)+\#(Y)$. This closes the induction. Hence, since both $X$ and $Y$ are finite,
        $X\cup Y$ is also finite.

        If $X$ and $Y$ are disjoint, by Lemma 3.6.9, we have 
        \begin{equation*}
            \begin{gathered}
                \#(X\cup Y-\{y_1\})=\#(X\cup Y)-1,\\
                \#((X\cup Y-\{y_1\}))=\#(X\cup Y-\{y_1\})-1,\\
                \vdots\\
                \#((X\cup Y-\{y_1\}-\cdots-\{y_{n-1}\})-\{y_n\})=\#(X\cup Y-\cdots-\{y_{n-1}\})-1.
            \end{gathered}
        \end{equation*}
        Sum these $n$ equations up, we have 
        \begin{equation*}
            \#(X)=\#((X\cup Y-\{y_1\}-\cdots-\{y_{n-1}\})-\{y_n\})=\#(X\cup Y)-\#(Y).
        \end{equation*}
        Thus, $\#(X\cup Y)=\#(X)+\#(Y)$.
    \end{proof}
    \item Let $X$ be a finite set, and let $Y$ be a subset of $X$. Then $Y$ is finite, and $\#(Y)\leq \#(X)$. If in addition $Y\ne X$, then we have $\#(Y)<\#(X)$.
    \begin{proof}
        Assume $Y\ne X$. Denote $X=\{x_1,\dotsc,x_n\}$, $Y=\{y_1,\dotsc, y_m\}$. Induct on $n$. When $n\leq m$, the statement is vacuously true. Suppose 
        inductively that $\#(Y)<\#(X)$ is true. Consider when $X=\{x_1,\dotsc,x_{n+1}\}$, $\#(\{x_1,\dotsc,x_{n+1}\})=\#(\{x_1,\dotsc,x_n\})+1$. So $\#(\{y_1,\dotsc,y_m\})< \#(\{x_1,\dotsc, x_n\})
        < \#(\{x_1,\dotsc, x_{n}\})+1=\#(\{x_1,\dotsc, x_{n+1}\})$. This closes the induction. For the case $Y=X$, $\#(Y)=\#(X)$, so $\#(Y)\leq \#(X)$. Since $\#(Y)\leq \#(X)$ and $X$ is finite, $Y$ is 
        also finite.
    \end{proof}
    \item If $X$ is a finite set, and $f:X\to Y$ is a function, then $f(X)$ is a finite set with $\#(f(X))\leq\#(X)$. If in addition $f$ is one-to-one, then $\#(f(X))=\#(X)$.
    \begin{proof}
        Denote $X=\{x_1,\dotsc,x_n\}$. Induct on $n$. When $n=0$, $\#f(X)=\#(X)=0$. The base case is proved. Suppose inductively $\#(f(X))\leq\#(X)$ is true for $n\in\mathbf{N}$. 
        Now consider $X=\{x_1,\dotsc,x_n,x_{n+1}\}$. By Lemma 3.6.9, $\#(\{x_1,\dotsc, x_{n+1}\})=\#(\{x_1,\dotsc, x_n\})+1$. By Proposition 3.6.14-(b),\\
         $f(\{x_1,\dotsc,x_n,x_{n+1}\})=f(\{x_1,\dotsc,x_n\})\cup f(x_{n+1})\leq \#f(\{x_1,\dotsc,x_n\})+1=\#(\{x_1,\dotsc,x_{n+1}\})$. This closes the induction.

         If $f$ is one-to-one, the proof is similar and we only need to modify a bit from the previous one. The proof of the base case stays the same. Suppose inductively $\#(f(X))=\#(X)$. Now 
         $f(\{x_1,\dotsc,x_{n+1}\})=f(\{x_1,\dotsc,x_n\})\cup f(x_{n+1})$, since $f$ is injective, these two sets are disjoint. By Proposition 3.6.14.(b), $\#(f\{x_1,\dotsc,x_{n+1}\})=\#(X)+1=\#(\{x_1,\dotsc,x_{n+1}\})$.
         This closes the induction.
    \end{proof}
    \item Let $X$ and $Y$ be finite sets. Then Cartesian product $X\times Y$ is finite and $\#(X\times Y)=\#(X)\times \#(Y)$.
    \begin{proof}
        Let $X$ has equal cardinality with $\{i\in\mathbf{N}:1\leq i\leq n\}$ and y has equal cardinality with $\{i\in\mathbf{N}:1\leq i\leq m\}$, use $f$ to denote this function. The statement we need to prove is $\#(X\times Y)$ 
        has equal cardinality with $\{i\in\mathbf{N}:1\leq i\leq nm\}$. Induct on $n$. When $n=0$, $\#(X\times Y)=\#(X)\times \#(Y)=0$. Suppose the statement is true for $n\in\mathbf{N}$. Consider when 
        $X$ has the same cardinality with $\{i\in\mathbf{N}:1\leq i\leq n+1\}$. By induction hypothesis, there exists a bijective function from $\#(X\times Y)$ to $\{i\in\mathbf{N}:1\leq i\leq nm\}$. Define the map from $X\times Y$ (partially)
        to $\{i\in\mathbf{N}:1\leq i\leq n+1\}$ as: for $x=x_{n+1}\in X$, $y\in Y$, $g(x,y)=nm+f(y)$. We need to verify that $g$ is also bijective. 

        For any $j\in\{i\in\mathbf{N}:1\leq i\leq nm\}$, by induction hypothesis, there exists a bijective function $h$ from $X\times Y$ to $\{i\in\mathbf{N}:1\leq i\leq nm\}$, so there exists some $x\in X$,
        $y\in Y$ such that $h(x,y)=j$. Let $g(x,y)=h(x,y)=j$ for $x\in X-\{x_{n+1}\}$ and $y\in Y$. For any $j\in\{i\in\mathbf{N}:nm+1\leq i\leq (n+1)m\}$, we have $g(n+1,j-nm)$. Thus, function $g$ is surjective. 
        Suppose $x_1,x_2\in X-\{x_{n+1}\}$, $y_1,y_2\in Y$, $(x_1,y_1)\ne(x_2,y_2)$, by induction hypothesis, $g(x_1,y_1)\ne g(x_2,y_2)$. 
        For $x_1\in X-\{x_{n+1}\}$, $x_2=x_{n+1}$, $y_1,y_2\in Y$, we have $g(x_1,y_1)\leq nm$ and $g(x_2,y_2)>nm$. So $g(x_1,y_1)\ne g(x_2,y_2)$. For $x_1=x_2=x_{n+1}$ and $y_1,y_2\in Y, y_1\ne y_2$, by definition, $g(x_1,y_1)\ne g(x_2,y_2)$. 
        Thus, $g$ is injective. So $g$ is bijective and thus $\#(X\times Y)$ has equal cardinality with $\{i\in\mathbf{N}:1\leq i\leq (n+1)m\}$.
    \end{proof}
    \item Let $X$ and $Y$ be finite sets. Then the set $Y^X$ is finite and $\#(Y^X)=\#(Y)^{\#(X)}$.
    \begin{proof}
        Denote $X=\{x_1,\dotsc,x_n\}$ and $Y=\{y_1,\dotsc,y_m\}$. The statement we need to prove is $Y^X$ has equal cardinality with $\{i\in\mathbf{N}:1\leq i\leq m^n\}$. Induct on $n$. When $n=0$, the number of functions from $X$ to the empty set is 1 which is equal 
        to $m^0$. This proved the base case. Suppose inductively $Y^X$ has equal cardinality with $\{i\in\mathbf{N}:1\leq i\leq m^n\}$ when $X=\{x_1,\dotsc,x_n\}$. Now consider when $X=\{x_1,\dotsc,x_{n+1}\}$. We want to show that it has equal cardinality with $M=\{i\in\mathbf{N}:1\leq i\leq m^{n+1}\}$.
        Define function $g$ that maps function $f$ such that $f(x_{n+1})=y_i\in Y$ to $(m^n+i)\in M$. The proof of the bijectivity of $g$ is similar to $(e)$. Once we have proved $g$ is bijective,
        the induction is closed. 
    \end{proof}
\end{enumerate}
\subsubsection*{Exercise 3.6.5}
Let $A$ and $B$ be sets. Show that $A\times B$ and $B\times A$ have equal cardinality by constructing an explicit bijection between the two sets. Then use Proposition 3.6.14 to conclude an alternate proof of Lemma 2.3.2.
\begin{proof}
    Firstly, we need to define a bijective function $f:A\times B\to B\times A$ such that for $a\in A$ and $b\in B$, $f(a,b)=(b,a)$. 
    Suppose $(a_1,b_1),(a_2,b_2)\in A\times B$, $f(a_1,b_1)=f(a_2,b_2)$. Then $(b_1,a_1)=(b_2,a_2)$. By definition of ordered pair, $a_1=a_2$ and $b_1=b_2$.
    Therefore, $(a_1,b_1)=(a_2,b_2)$. Thus, $f$ is injective. For any $(b,a)\in B\times A$, there exists $(a,b)\in A\times B$ such that $f(a,b)=(b,a)$. So $f$ is surjective.
    Hence, $f$ is bijective, and $A\times B$ and $B\times A$ have equal cardinality. Suppose $\#(A)=n$ and $\#(B)=m$. By Proposition 3.6.14,
    \begin{equation*}
        \begin{gathered}
            \#(A\times B)=\#(A)\times \#(B)=n\times m,\\
            \#(B\times A)=\#(B)\times \#(A)=m\times n,
        \end{gathered}
    \end{equation*}
    and since 
    \begin{equation*}
        \#(A\times B)=\#(B\times A),
    \end{equation*}
    we have 
    \begin{equation*}
        n\times m=m\times n.
    \end{equation*}
\end{proof} 
\subsubsection*{Exercise 3.6.6}
Let $A,B,C$ be sets. Show that the sets $(A^B)^C$ and $A^{B\times C}$ have equal cardinality by constructing an explicit bijection between the two sets.
Conclude that $(a^b)^c=a^{bc}$ for any natural numbers $a,b,c$. Use similar argument to also conclude $a^b\times a^c=a^{b+c}$.
\begin{proof}
    Suppose we have $f:B\to A$, $g:C\to A^B$ and $l:B\times C\to A$, let $h:(A^B)^C\to A^{B\times C}$ be a function such that for $g\in (A^B)^C$, for all $b\in B, c\in C$,
    $(h(g))(b,c)=(g(c))(b)$. Suppose $g_1,g_2\in (A^B)^C$ and $h(g_1)=h(g_2)$. Then by definition, for any $b\in B$, $c\in C$, $(g_1(c))(b)=(g_2(c))(b)$. Let $g_1(c)=f_1$ and $g_2(c)=f_2$. 
    Since for all $b\in B$, $f_1(b)=f_2(b)$, we have $f_1=f_2$. Then for all $c\in C$, $f_1=g_1(c)=g_2(c)=f_2$, so $g_1=g_2$. Thus, $h$ is injective. For any function $l$, for any
    $b\in B$, $c\in C$, let $f(b)=l(b,c)$ and $g(c)=f$, by definition, we have $(h(g))(b,c)=l$. Thus, $h$ is surjective. Since $h$ is both injective and surjective, $h$ is bijective. Thus,
    $(A^B)^C$ and $A^{B\times C}$ have equal cardinality. 
    
    Suppose $\#(A)=a$, $\#(B)=b$ and $\#(C)=c$. By Proposition 3.6.14,
    \begin{equation*}
        \begin{gathered}
            \#(A^B)^C=\#(A^B)^{\#(C)}=(\#(A)^{\#(B)})^{\#(C)}=(a^b)^c,\\
            \#(A^{B\times C})=\#(A)^{\#(B\times C)}=\#(A)^{\#(B)\times \#(C)}=a^{b\times c},
        \end{gathered}
    \end{equation*}
    since $\#((A^B)^C)=\#(A^{B \times C})$, we have 
    \begin{equation*}
        (a^b)^c=a^{b\times c}.
    \end{equation*}
    Suppose $f: B\to A$, $g:C\to A$, $l:B\cup C\to A$ and $B$ and $C$ are disjoint. Define $h:A^B\times A^C\to A^{B\cup C}$ as for $b\in B$,
    $c\in C$, $(h(f,g))(b)=f(b)$ and $(h(f,g))(c)=g(c)$. We can show that $h$ is bijective in a similar way. Use this argument and Proposition 3.6.14
    denote $\#(A)=a,\#(B)=b,\#(C)=c$, we can show that $a^b\times a^c=a^{b+c}$.
\end{proof}
\subsubsection*{Exercise 3.6.7}
Let $A$ and $B$ be sets. Let us say that $A$ has lesser or equal cardinality to $B$ if there exists an injection $f:A\to B$ from $A$ to $B$.
Show that if $A$ and $B$ are finite sets, then $A$ has lesser or equal cardinality to $B$ iff $\#(A)\leq\#(B)$.
\begin{proof}
    We need to show that $\exists$ injection $f:A\to B\iff \#(A)\leq \#(B)$.

    Suppose $\exists$ injection $f:A\to B$. By Proposition 3.6.14, $\#(f(A))=\#(A)$. By definition of function, if $a\in A$, $f(a)\in B$. So $f(A)\subseteq B$.
    Again by Proposition 3.6.14, $\#f(A)\leq \#f(B)$. Hence, $\#(A)=\#f(A)\leq\#(B)$.

    Suppose $\#(A)\leq \#(B)$. Let $\#(A)=m$ and $\#(B)=n$. By definition, there exists a bijection $f:A\to \{i\in\mathbf{N}:1\leq i\leq m\}$ and 
    a bijection $g:B\to \{i\in\mathbf{N}:1\leq i\leq n\}$. As we have shown in Exercise 3.6.1, $g^{-1}$ is also bijective. Consider $h=g^{-1}\circ f$. $\{i\in\mathbf{N}:1\leq i\leq m\}
    \in\{i\in\mathbf{N}:1\leq i\leq n\}$, so the range of $f$ is within the domain of $g^{-1}$. Then $h$ is a valid function from $A$ to $B$. 
    We need to show that $h$ is injective. Suppose we have $a_1,a_2\in A$ such that $g^{-1}(f(a_1))=g^{-1}(f(a_2))$. Since $g^{-1}$ is bijective, there must be $f(a_1)=f(a_2)$.
    Since $f$ is bijective, we have $a_1=a_2$. Thus, $h$ is injective. By definition, $A$ has lesser or equal cardinality to $B$.
\end{proof}
\subsubsection*{Exercise 3.6.8}
Let $A$ and $B$ be sets such that there exists an injection $f:A\to B$ from $A$ to $B$. Assume also that $A$ is non-empty. Show that there exists a surjection $g:B\to A$ from $B$
to $A$. 
\begin{proof}
    Define $g:B\to A$ as below 
    \begin{equation*}
        g(b)=
        \begin{cases}
            a \text{ such that } f(a)=b, & b\in f(A)\\
            0, & \text{otherwise}
        \end{cases}.
    \end{equation*}
    Then for any $a\in A$, we have $f(a)\in B$ such that $g(f(a))=a$. Hence, $g$ is surjective.
\end{proof}
\subsubsection*{Exercise 3.6.9}
Let $A$ and $B$ be finite sets. Show that $A\cup B$ and $A\cap B$ are also finite sets, and that $\#(A)+\#(B)=\#(A\cup B)+\#(A\cap B)$.
\begin{proof}
    By Exercise 3.1.10, $A\cup B=(A\backslash B)\cup (B\backslash A)\cup (A\cap B)$ and $A\backslash B$, $B\backslash A$, and $A\cap B$ are disjoint. 
    $A=A\backslash B+A\cap B \implies \#(A)=\#(A\backslash B)+\#(A\cap B)$. $B=B\backslash A+A\cap B \implies \#(B)=\#(B\backslash A)+\#(A\cap B)$.
    So $\#(A)+\#(B)=(\#(A\backslash B)+\#(B\backslash A)+\#(A\cap B))+\#(A\cap B)=\#(A\cap B)+\#(A\cap B)$.
\end{proof}
\subsubsection*{Exercise 3.6.10}
Let $A_1, \dotsc, A_n$ be finite sets such that $\#(\bigcup_{i\in\{1,\dotsc,n\}}A_i)>n$. Show that there exists
$i\in\{1,\dotsc,n\}$ such that $\#(A_i)\geq 2$. (This is known as the pigeonhole principle.)
\begin{proof}
    Suppose for all $i\in\{1,\dotsc,n\}$, $\#(A_i)\leq 1$. Then $\#(\bigcup_{i\in\{1,\dotsc,n\}}A_i)=\#(A_1\cup\dotsc\cup A_n)\leq 
    \#(A_1)+\cdots+\#(A_n)\leq n$. (Contradiction.) Therefore, there exists $i\in\{1,\dotsc,n\}$ such that $\#(A)\geq 2$.
\end{proof}
\end{document}