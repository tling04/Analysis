\documentclass[12pt, letter]{article}
\usepackage[utf8]{inputenc}
\usepackage[a4paper, total={6in, 8in}]{geometry}
\usepackage{tikz}
\usepackage[T1]{fontenc}
\usepackage{listings}
\usepackage{graphicx}
\usepackage{amsfonts}
\usepackage{amsmath}
\usepackage{amssymb}
\usepackage{amsthm}
\usepackage{mathtools}
\usepackage{listings}
\usepackage{bm}
\newcommand{\uvec}[1]{\boldsymbol{\hat{\textbf{#1}}}}
\usepackage[english]{babel}
\newtheorem{theorem}{Theorem}
\usepackage{setspace}

\setstretch{1.25}
\begin{document}
\section*{Chapter 3}
\section*{Set Theory}
\subsubsection*{Definition 3.1.1}
(Informal) We define a $set$ $A$ to be any unordered collection of objects, e.g., ${3,8,5,2}$ is a set. If $x$ is an object, we say that $x$ is an element of $A$ or $x\in A$ if $x$ lies in the collection;
otherwise we say that $x\notin A$. For instance, $3\in\{1,2,3,4,5\}$ but $7\notin \{1,2,3,4,5\}$.
\subsubsection*{Axiom 3.1 (Sets are objects).} 
If $A$ is a set, then $A$ is also an object. In particular, given two sets $A$ and $B$, it is meaningful to ask whether $A$ is also an element of $B$.
\subsubsection*{Axiom 3.2 (Equality of sets).}
Two sets $A$ and $B$ are equal, $A=B$, iff every element of $A$ is an element of $B$ and vice versa. To put it another way, $A=B$ if and only if every element $x$ of $A$ belongs also to $B$, and every element $y$
of $B$ belongs also to $A$.
\subsubsection*{Axiom 3.3 (Empty set).}
There exists a set $\emptyset$, known as the empty set, which contains no elements, i.e., for every object $x$ we have $x\notin\emptyset$. 
\subsubsection*{Lemma 3.1.5 (Single choice).}
Let $A$ be a non-empty set. Then there exists an object $x$ such that $x\in A$. 
\subsubsection*{Axiom 3.4 (Singleton sets and pair sets).}
If $a$ is an object, then there exists a set $\{a\}$ whose only element is $a$, i.e., for every object $y$, we have $y\in\{a\}$ if and only if $y=a$; we refer to $\{a\}$ as the singleton set 
whose element is $a$. Furthermore, if $a$ and $b$ are objects, then there exists a set $\{a, b\}$ whose only elements are $a$ and $b$; i.e., for every object $y$, we have $y\in\{a,b\}$ if and only if $y=a$ or $y=b$; 
we refer to this set as the pair set formed by $a$ and $b$.
\subsubsection*{Axiom 3.5 (Pairwise union).}
Given any two sets $A,B$, there exists a set $A\cup B$, called the union of $A$ and $B$, which consists of all the elements which belong to $A$ or $B$ or both. In other words, for any object $x$,
\begin{equation*}
    x\in A\cup B\iff (x\in A \text{ or } x\in B).
\end{equation*}
\subsubsection*{Lemma 3.1.12}
If $a$ and $b$ are objects, then $\{a,b\}=\{a\}\cup\{b\}$. If $A,B,C$ are sets, then the union operation is commutative (i.e., $A\cup B=B\cup A$) and associative (i.e., $(A\cup B)\cup C=A\cup (B\cup C)$). Also,
we have $A\cup A=A\cup\emptyset=\emptyset\cup A=A$.
\subsubsection*{Definition 3.1.14 (Subsets).}
Let $A,B$ be sets. We say that $A$ is a subset of $B$, denoted $A\subseteq B$, iff every element of $A$ is also an element of $B$, i.e.
\begin{equation*}
    \text{For any object }x,\; x\in A\iff x\in B,
\end{equation*} 
We say that $A$ is a proper subset of $B$, denoted $A\subsetneq B$, if $A\subseteq B$ and $A\ne B$.
\subsubsection*{Proposition 3.1.17 (Sets are partially ordered by set inclusion).}
Let $A,B,C$ be sets. If $A\subseteq B$ and $B\subseteq C$ then $A\subseteq C$. If $A\subseteq B$ and $B\subseteq A$, then $A=B$. Finally, if $A\subsetneq B$ and $B\subsetneq C$ then $A\subsetneq C$.



\subsection*{Exercises}
\subsubsection*{Exercise 3.1.1}
Let $a,b,c,d$ be objects such that $\{a,b\}=\{c,d\}$. Show that at least one of the two statements "$a=c$ and $b=d$" and "$a=d$ and $b=c$" hold.
\begin{proof}
    Consider two cases: $a=b$ and $a\ne b$.\\
    Case 1: $a=b$. Then $\{a,b\}=\{a\}$. By Axiom 3.2, if $\{a\}$ and $\{c,d\}$ are equal to each other, then every element belong to $\{c,d\}$ must also belong to $\{a\}$. 
    Therefore, $c=a$, $d=a$. Since $a=b$, we have $a=b=c=d$. Thus, both statements hold.\\
    Case 2: $a\ne b$. Similarly, by Axiom 3.2, every element belong to $\{a,b\}$ must also belong to $\{c,d\}$. So $\{c,d\}$, a set of two elements, contains two distinct elements $a$ and $b$. Therefore, 
    either $a=c, b=d$ or $a=d, b=c$ holds, exclusively.\\
    Thus, we have shown that at least one of the two statements "$a=c$ and $b=d$" and "$a=d$ and $b=c$" hold.
\end{proof}
\subsubsection*{Exercise 3.1.2}
Using only Axiom 3.2, Axiom 3.1, Axiom 3.3, and Axiom 3.4, prove that the sets $\emptyset$, $\{\emptyset\}$, $\{\{\emptyset\}\}$, and $\{\emptyset,\{\emptyset\}\}$ are all distinct.
\begin{proof}
    First, let's consider $\emptyset$. $\emptyset$ contains no element while other sets all have at least one element in it. Therefore, $\emptyset$ is distinct from $\{\emptyset\}$, $\{\{\emptyset\}\}$ and $\{\emptyset,\{\emptyset\}\}$. 
    Then, let's consider $\{\emptyset\}$. Is it distinct from $\{\{\emptyset\}\}$ and $\{\emptyset,\{\emptyset\}\}$?
    We know that $\emptyset\in\{\emptyset\}$. But we have proved earlier $\emptyset$ and $\{\emptyset\}$ are not equal to each other, so $\emptyset\notin \{\{\emptyset\}\}$. So $\{\emptyset\}$ and $\{\{\emptyset\}\}$ are distinct. For the same reason, 
    $\{\emptyset\}\notin \{\emptyset\}$. So $\{\emptyset\}$ and $\{\emptyset,\{\emptyset\}\}$ are also distinct. Last, consider $\{\{\emptyset\}\}$ and $\{\emptyset,\{\emptyset\}\}$. For the same reason ($\emptyset$ and $\{\emptyset\}$ are distinct), 
    $\emptyset\notin \{\{\emptyset\}\}$. So $\{\{\emptyset\}\}$ and $\{\emptyset,\{\emptyset\}\}$ are distinct. Thus, we have proved the sets $\emptyset$, $\{\emptyset\}$, $\{\{\emptyset\}\}$, and $\{\emptyset,\{\emptyset\}\}$ are all distinct.
\end{proof}
\subsubsection*{Exercise 3.1.3}
Prove the remaining claims in Lemma 3.1.12.
\begin{proof}
    First, prove the union operation is commutative (i.e., $A\cup B=B\cup A$). By definition, we know that $A\cup B$ consists of all the elements which belong to $A$ or $B$, inclusively. 
    And $B\cup A$ also consists of all the elements belong to $A$ or $B$, inclusively. Therefore, $A\cup B$ and $B\cup A$ are containing exactly the same elements. Thus, $A\cup B=B\cup A$.
    
    The second part is to prove $A\cup A=A\cup\emptyset=\emptyset\cup A=A$. First, let's consider $A\cup A$. By definition, $A\cup A$ consists of all the element $x$ such that $x\in A$ or $x\in A$. So $A\cup A$ and $A$ have exactly the same elements. Therefore $A\cup A=A$.
    Now let's consider $A\cup\emptyset$. $A\cup\emptyset$ consists of all the $x$ such that $x\in A$ or $x\in\emptyset$. Since no element would belong to $\emptyset$. $A\cup\emptyset$ contains exactly the same elements as $A$. Therefore, $A\cup\emptyset=A$. And by commutative law, 
    we have $A\cup\emptyset=\emptyset\cup A=A$.

    Thus, we have proved $A\cup A=A\cup\emptyset=\emptyset\cup A=A$.
\end{proof}
\subsubsection*{Exercise 3.1.4}
Prove the remaining claims in Lemma 3.1.17.
\begin{proof}
    Part I. If $A\subseteq B$ and $B\subseteq A$, then $A=B$. Translate the if statement into propositional logic: $(x\in A \Rightarrow x\in B) \wedge (x\in B\Rightarrow x\in A)$. Therefore, we have $x\in A\iff x\in B$. Thus, $A=B$.

    Part II. If $A\subsetneq B$ and $B\subsetneq C$ then $A\subsetneq C$. Since $A\ne B$ and $B\ne C$, by transitivity, $A\ne C$. And by the first part of this proposition (if $A\subseteq B$ and $B\subseteq C$ then $A\subseteq C$), we would have $A\subseteq C$. Since $A\subseteq C$ and $A\ne C$, $A\subsetneq C$.
\end{proof}
\end{document}