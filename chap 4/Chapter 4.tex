\documentclass[12pt, letter]{article}
\usepackage[utf8]{inputenc}
\usepackage[a4paper, total={6in, 8in}]{geometry}
\usepackage{tikz}
\usepackage[T1]{fontenc}
\usepackage{listings}
\usepackage{graphicx}
\usepackage{amsfonts}
\usepackage{amsmath}
\usepackage{amssymb}
\usepackage{enumitem}
\usepackage{amsthm}
\usepackage{mathtools}
\usepackage{listings}
\usepackage{bm}
\newcommand{\uvec}[1]{\boldsymbol{\hat{\textbf{#1}}}}
\usepackage[english]{babel}
\newtheorem{theorem}{Theorem}
\usepackage{setspace}

\setstretch{1.25}
\begin{document}
\section*{Chapter 4}
\section*{Integers and rationals}
\subsection*{4.1 The integers}
\subsubsection*{Definition 4.1.1 (Integers).}
An integer is an expression of the form $a--b$, where $a$ and $b$ are natural numbers. Two integers are considered to be equal, 
$a--b=c--d$, if and only if $a+d=c+b$. We let $\mathbf{Z}$ denote the set of all integers.
\subsubsection*{Definition 4.1.2}
The sum of two integers, $(a--b)+(c--d)$, is defined by the formula
\begin{equation*}
    (a--b)+(c--d):=(a+c)-(b+d).
\end{equation*}
The product of two integers, $(a--b)\times (c--d)$, is defined by
\begin{equation*}
    (a--b)\times(c--d):=(ac+bd)--(ad+bc).
\end{equation*}
\subsubsection*{Lemma 4.1.3 (Addition and multiplication are well-defined).}
Let $a,b,a',b',c,d$ be natural numbers. If $(a--b)=(a'--b')$, then $(a--b)+(c--d)=(a'--b')+(c--d)$ and 
$(a--d)\times(c--d)=(a'--b')\times(c--d)$, and also $(c--d)+(a--b)=(c--d)+(a'--b')$ and $(c--d)\times(a--b)=(c--d)\times(a'--b')$.
Thus addition and multiplication are well-defined operations (equal inputs give equal outputs).
\subsubsection*{Definition 4.1.4 (Negation of integers).}
If $(a--b)$ is an integer, we define the negation $-(a--b)$ to be the integer $(b--a)$. In particular if $n=n--0$
is a positive natural number, we can define its negation $-n=0--n$.
\subsubsection*{Lemma 4.1.5 (Trichotomy of integers).}
Let $x$ be an integer. Then exactly one of the following three statements is true: (a) $x$ is zero;
(b) $x$ is equal to a positive natural number $n$; or (c) $x$ is the negation $-n$ of a positive natural number $n$.
\subsubsection*{Proposition 4.1.6 (Laws of algebra for integers).}
Let $x,y,z$ be integers. Then we have 
\begin{equation*}
    \begin{aligned}
        x+y&=y+x\\
        (x+y)+z&=x+(y+z)\\
        x+0=0+x&=x\\
        x+(-x)=(-x)+x&=0\\
        xy&=yx\\
        (xy)z&=x(yz)\\
        x1=1x&=x\\
        x(y+z)&=xy+xz\\
        (y+z)x&=yx+zx
    \end{aligned}
\end{equation*}
\subsubsection*{Proposition 4.1.8 (Integers have no zero divisors).}
Let $a$ and $b$ be integers such that $ab=0$. Then either $a=0$ or $b=0$ (or both).
\subsubsection*{Corollary 4.1.9 (Cancellation law for integers).}
If $a,b,c$ are integers such that $ac=bc$ and $c$ is non-zero, then $a=b$.
\subsubsection*{Definition 4.1.10 (Ordering of the integers).}
If $n$ and $m$ be integers. We say that $n$ is greater than or equal to $m$, and write $n\geq m$ or $m\leq n$,
iff we have $n=m+a$ for some natural number $a$. We say that $n$ is strictly greater than $m$,
and write $n>m$ or $m<n$, iff $n\geq m$ and $n\ne m$.
\subsubsection*{Lemma 4.1.11 (Properties of order).}
Let $a,b,c$ be integers.
\begin{enumerate}[label=(\alph*)]
    \item $a>b$ if and only if $a-b$ is a positive natural number.
    \item (Addition preserves order) If $a>b$, then $a+c>b+c$.
    \item (Positive multiplication preserves order) If $a>b$ and $c$ is positive, then $ac>bc$.
    \item (Negation reverses order) If $a>b$ and $b>c$, then $a>c$.
    \item (Order trichotomy) Exactly one of the statements $a>b$, $a<b$, or $a=b$ is true.
\end{enumerate}
\subsubsection*{Exercise 4.1.1}
Verify that the definition of equality on the integers is both reflexive and symmetric.
\begin{proof}
    Reflexivity: since summation is reflexive, we have $a+b=a+b$. Thus, by definition, $a--b=a--b$. Symmetry: assume $a--b=c--d$, then $a+d=c+b$.
    Since summation is symmetric, $c+b=a+d$. By definition, we have $c--d=a--b$.
\end{proof}
\subsubsection*{Exercise 4.1.2}
Show that the definition of negation on the integers is well-defined in the sense that $(a--b)=(a'--b')$, then $-(a--b)=-(a'--b')$ (so equal integers have equal negations).
\begin{proof}
    Since $(a--b)=(a'--b')$, by definition, $a+b'=a'+b$. By the reflexivity and symmetry of summation, we have $b+a'=b'+a$. Thus, by definition, 
    $b--a=b'--a'$. By definition of negation of integers, $-(a--b)=-(a'--b')$.
\end{proof}
\subsubsection*{Exercise 4.1.3}
Show that $(-1)\times a=-a$ for every integer $a$.
\begin{proof}
    By definition, $-1=(0--1)$ and $a=(a--0)$. Then $(-1)\times a=(0--1)\times (a--0)=(0\times a+1\times 0)--(0\times 0+1\times a)=0--a=-a$.
\end{proof}
\subsubsection*{Exercise 4.1.4}
Prove the remaining identities in Proposition 4.1.6.
\begin{enumerate}
    \item $x+y=y+x$.
    \begin{proof}
        Suppose $x=a--b$ and $y=c--d$ for some natural numbers $a,b,c,d$. Then $x+y=(a--b)+(c--d)=(a+c)--(b+d)$ and $y+x=(c--d)+(a--b)=(c+a)--(d+b)$. By the symmetry property of summation,
        we have $(a+c)=(c+a)$ and $(b+d)=(d+b)$. Thus, $x+y=y+x$.
    \end{proof}
    \item $(x+y)+z=x+(y+z)$.
    \begin{proof}
        Suppose $x=a--b$, $y=c--d$, and $z=e--f$ for some natural numbers $a,b,c,d,e,f$,. Then 
        \begin{equation*}
            \begin{aligned}
                (x+y)+z&=((a--b)+(c--d))+(e--f)\\
                &=((a+c)--(b+d))+(e--f)\\
                &=((a+c)+e)--((b+d)+f)\\
                &=(a+c+e)--(b+d+f);\\
                x+(y+z)&=(a--b)+((c--d)+(e--f))\\
                &=(a--b)+((c+e)--(d+f))\\
                &=(a+(c+e))--(b+(d+f))\\
                &=(a+c+e)--(b+d+f).
            \end{aligned}
        \end{equation*}
        Therefore, $(x+y)+z=x+(y+z)$.
    \end{proof}
    \item $x+0=0+x=x$.
    \begin{proof}
        Since $x+y=y+x$, we have $x+0=0+x$. Let $x=a--b$ for some natural numbers $a,b$, and write $0=0--0$. Then $x+0=(a--b)+(0--0)=(a+0)--(b+0)=a--b=x$. 
        Thus, $x+0=0+x=x$.
    \end{proof}
    \item $x+(-x)=(-x)+x=0$.
    \begin{proof}
        Since $x+y=y+x$, we have $x+(-x)=(-x)+x$. Let $x=a--b$ for some natural numbers $a,b$, then $-x=b--a$. Write $0$ as $0--0$. 
        Then $x+(-x)=(a--b)+(b--a)=(a+b)--(b+a)$. Since $(a+b)+0=(b+a)+0=a+b$, we have that $(a+b)--(b+a)=0--0$. So $x+(-x)=0$. Thus, $x+(-x)=(-x)+x=0$.
    \end{proof}
    \item $xy=yx$.
    \begin{proof}
        Let $x=a--b$ and $y=c--d$ for some natural numbers $a,b,c,d$. Then 
        \begin{equation*}
            \begin{aligned}
                xy&=(a--b)\times (c--d)\\
                &=(ac+bd)--(ad+bc);\\
                yx&=(c--d)\times (a--b)\\
                &=(ca+db)--(cb+da)\\
                &=(ac+bd)--(ad+bc).
            \end{aligned}
        \end{equation*}
        Therefore, $xy=yx$.
    \end{proof}
    \item $(xy)z=x(yz)$.\\
    Has been proved on page 79.
    \item $x1=1x=x$.
    \begin{proof}
        Since $xy=yx$, we have $x1=1x$. Let $x=a--b$ for some natural numbers $a,b$. $1x=(1--0)(a--b)=1a--1b=a--b=x$. Thus, $x1=1x=x$.
    \end{proof}
    \item $x(y+z)=xy+xz$.
    \begin{proof}
        Let $x=a--b$, $y=c--d$, and $z=e--f$ for some natural numbers $a,b,c,d,e,f$. Then 
        \begin{equation*}
            \begin{aligned}
                x(y+z)&=(a--b)((c--d)+(e--f))\\
                &=(a--b)((c+e)--(d+f))\\
                &=(a(c+e)+b(d+f))--(a(e+f)+b(c+d))\\
                &=(ac+ae+bd+bf)--(ae+af+bc+bd);\\
                xy+xz&=(a--b)(c--d)+(a--b)(e--f)\\
                &=((ac+bd)--(ad+bc))+((ae+bf)--(af+be))\\
                &=((ac+bd)+(ae+bf))--((ad+bc)+(af+be))\\
                &=(ac+ae+bd+bf)--(ae+af+bc+bd).
            \end{aligned}
        \end{equation*}
        Therefore, $x(y+z)=xy+xz$.
    \end{proof}
    \item $(y+z)x=yx+zx$.
    \begin{proof}
        Since $xy=yx$, we have $(y+z)x=x(y+z)$. By using identities, we get $xy+xz=yx+zx$. And because $x(y+z)=xy+xz$, $(y+z)x=yx+zx$.
    \end{proof}
\end{enumerate}
\subsubsection*{Exercise 4.1.5}
Prove Proposition 4.1.8.
\begin{proof}
    From now on we could just use $-$ instead of $--$. Let $a=c-d$ and $b=e-f$ for some natural numbers $c,d,e,f$. So $ab=0\implies (c-d)(e-f)=0$.
    Assume $c-d\geq 0$ and $e-f\geq 0$, by Lemma 2.3.3, at least one of $a=(c-d)$ and $b=e-f$ is equal to 0. If at least one of $(c-d)$ and $(e-f)$ 
    is negative, without loss of generality, assume $c-d<0$. Then $-(c-d)=d-c>0$ and we have $(d-c)(e-f)=-1\times 0=0$. By Lemma 2.3.3, 
    at least one of $-a=d-c$ and $b=e-f$ is equal to zero, and this statement is equivalent to either $a=0$ or $b=0$ (or both). 
\end{proof}
\subsubsection*{Exercise 4.1.6}
Prove Corollary 4.1.9.
\begin{proof}
    $ac=bc\implies ac-bc=ac+(-b)c=(a+(-b))c=(a-b)c=0$ by Proposition 4.1.6. By Proposition 4.1.8, at least one of $(a-b)$ and $c$ is equal to 0. Since 
    $c\ne 0$, $a-b=0$. Thus, $a=b$.
\end{proof}
\subsubsection*{Exercise 4.1.7}
Prove Lemma 4.1.11.
\begin{enumerate}[label=(\alph*)]
    \item \begin{proof}
        We need to show that $a>b\iff a-b$ is a positive natural number. Suppose $a>b$. By definition, there exists a positive natural number $n$ such that 
        $a=b+n$. So $a-b=n>0$ as required. Suppose $a-b$ is a positive natural number. Then $a-b=n\iff a=b+n$ for some positive integer $n$. Thus, $a>b$.
    \end{proof}
    \item \begin{proof}
        Since $a>b$, there exists a positive natural number $n$ such that $a-b=n$. Then $(a+c)-b=c+n\iff (a+c)=(b+c)+n$. Therefore, $a+c>b+c$.
    \end{proof} 
    \item \begin{proof}
        Since $a>b$, there exists a positive natural number $n$ such that $a-b=n$. Since $(a-b)$ and $n$ are both natural numbers, we have $c(a-b)=cn$ for any positive integer $c$.
        Then $ac=bc+cn$ where $cn$ is a positive natural number. Therefore, $ac>bc$.
    \end{proof}
    \item \begin{proof}
        Since $a>b$, there exists a positive natural number $n$ such that $a-b=n$. Then $-b=-a+n$. Since $n>0$, $-b>-a$.
    \end{proof}
    \item \begin{proof}
        Since $a>b$, there exists a positive natural number $n$ such that $a-b=n$. Since $b>c$, there exists a positive natural number $m$ such that $b-c=m$.
        Then $(a-b)+(b-c)=a-c=n+m$ where $(n+m)$ is a positive natural number. Therefore, $a>c$. 
    \end{proof}
    \item \begin{proof}
        Since $a-b$ is an integer, by Lemma $4.1.5$, exactly one of the following three statement is true:
        \begin{enumerate}[label=(\alph*)]
            \item $a-b$ is zero. Then $a=b$.
            \item $a-b$ is equal to a positive natural number $n$. $a-b=n$, so $a>b$.
            \item $-(a-b)=b-a$ is equal to a positive natural number $n$. $b-a=n$, so $b>a$ which is equivalent to $a<b$. 
        \end{enumerate}
    \end{proof}
\end{enumerate}
\subsubsection*{Exercise 4.1.8}
Show that the principle of induction does not apply directly to the integers. More precisely, give an example of a property $P(n)$ pertaining to an integer $n$
such that $P(0)$ is true, and that $P(n)$ implies $P(n++)$ for all integers $n$, but that $P(n)$ is not true for all integers $n$. Thus induction is not as useful a tool for
dealing with the integers as it is with the natural numbers.
\begin{proof}
    A counterexample of $P(n)$ could be $f(n)=n^2$ is a monotonically increasing function. 
\end{proof}
\end{document}