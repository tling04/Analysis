\documentclass[12pt, letter]{article}
\usepackage[utf8]{inputenc}
\usepackage[a4paper, total={6in, 8in}]{geometry}
\usepackage{tikz}
\usepackage[T1]{fontenc}
\usepackage{listings}
\usepackage{graphicx}
\usepackage{amsfonts}
\usepackage{amsmath}
\usepackage{amssymb}
\usepackage{amsthm}
\usepackage{mathtools}
\usepackage{enumitem}
\usepackage{listings}
\usepackage{bm}
\newcommand{\uvec}[1]{\boldsymbol{\hat{\textbf{#1}}}}
\newcommand{\la}{\lim_{n\to\infty}a_n}
\newcommand{\lb}{\lim_{n\to\infty}b_n}
\newcommand{\an}{(a_n)_{n=m}^\infty}
\newcommand{\bn}{(b_n)_{n=m}^\infty}
\usepackage[english]{babel}
\newtheorem{theorem}{Theorem}
\usepackage{setspace}

\setstretch{1.25}
\begin{document}
\section*{Chapter 6}
\section*{Limits of sequences}
\subsection*{6.1 Convergence and limit laws}
\subsubsection*{Definition 6.1.1 (Distance between two real numbers).}
Given two real numbers $x$ and $y$, we define their distance $d(x,y)$ to be $d(x,y):=|x-y|$.
\subsubsection*{Definition 6.1.2 ($\varepsilon$-close real numbers).}
Let $\varepsilon>0$ be a real number. We say that two real numbers $x,y$ are $\varepsilon$-close iff we have $d(y,x)\leq \varepsilon$.
\subsubsection*{Definition 6.1.3 (Cauchy sequences of reals).}
Let $\varepsilon>0$ be a real number. A sequence $(a_n)_{n=N}^\infty$ of real numbers starting at some integer index $N$ is said to be $\varepsilon$-steady
iff $a_j$ and $a_k$ are $\varepsilon$-close for every $j,k\geq N$. A sequence $(a_n)_{n=m}^\infty$ starting at some integer index $m$ is said to be
eventually $\varepsilon$-steady iff there exists an $N\geq m$ such that $(a_n)_{n=N}^\infty$ is $\varepsilon$-steady. We say that $(a_n)_{n=m}^\infty$
is a Cauchy sequence iff it is eventually $\varepsilon$-steady for every $\varepsilon>0$. 
\subsubsection*{Proposition 6.1.4.}
Let $(a_n)_{n=m}^\infty$ be a sequence of rational numbers starting at some integer index $m$. Then $(a_n)_{n=m}^\infty$ is a Cauchy sequence 
in the sense of Definition 5.1.8 if and only if it is a Cauchy sequence in the sense of Definition 6.1.3.
\subsubsection*{Definition 6.1.5 (Convergence of sequences).}
Let $\varepsilon>0$ be a real number, and let $L$ be a real number. A sequence $(a_n)_{n=N}^\infty$ of real numbers is said to be $\varepsilon$-close
to $L$ iff $a_n$ is $\varepsilon$-close to $L$ for every $n\geq N$, i.e., we have $|a_n-L|\leq \varepsilon$ for every $n\geq N$. We say that a sequence $(a_n)_{n=m}^\infty$
is eventually $\varepsilon$-close to $L$ iff there exists an $N\geq m$ such that $(a_n)_{n=N}^\infty$ is $\varepsilon$-close to $L$. We say that a sequence
$(a_n)_{n=m}^\infty$ converges to $L$ iff it is eventually $\varepsilon$-close to $L$ for every real $\varepsilon>0$.
\subsubsection*{Proposition 6.1.7 (Uniqueness of limits).}
Let $(a_n)_{n=m}^\infty$ be a real sequence starting at some integer index $m$, and let $L=L'$ be two distinct real numbers. Then it is not possible for $(a_n)_{n=m}^\infty$
to converge to $L$ while also converging to $L'$.
\subsubsection*{Definition 6.1.8 (Limits of sequences).}
If a sequence $(a_n)_{n=m}^\infty$ converges to some real number $L$, we say that $(a_n)_{n=m}^\infty$ is convergent and that its limit is $L$; we write
\begin{equation*}
    L=\lim_{n\to\infty}a_n
\end{equation*}
to denote this fact. If a sequence $(a_n)_{n=m}^\infty$ is not converging to any real number $L$, we say that the sequence $(a_n)_{n=m}^\infty$ is divergent 
and we leave $\lim_{n\to\infty}a_n$ undefined.
\subsubsection*{Proposition 6.1.11.}
We have $\lim_{n\to\infty}1/n=0$.
\subsubsection*{Proposition 6.1.12 (Convergent sequences are Cauchy).}
Suppose that $(a_n)_{n=m}^\infty$ is a convergent sequence of real numbers. Then $(a_n)_{n=m}^\infty$ is also a Cauchy sequence.
\subsubsection*{Proposition 6.1.15 (Formal limits are genuine limits).}
Suppose that $(a_n)_{n=1}^\infty$ is a Cauchy sequence of rational numbers. Then $(a_n)_{n=1}^\infty$ converges to LIM$_{n\to\infty}a_n$, i.e.,
\begin{equation*}
    \text{LIM}_{n\to\infty}a_n=\lim_{n\to\infty}a_n. 
\end{equation*}
\subsubsection*{Definition 6.1.16 (Bounded sequences).}
A sequence $(a_n)_{n=m}^\infty$ of real numbers is bounded by a real number $M$ iff we have $|a_n|\leq M$ for all $n\geq m$. We say that $(a_n)_{n=m}^\infty$ is bounded 
iff it is bounded by $M$ for some real number $M>0$.
\subsubsection*{Corollary 6.1.17.}
Every convergent sequence of real numbers is bounded.
\subsubsection*{Theorem 6.1.19 (Limit Laws).}
Let $(a_n)_{n=m}^\infty$ and $(b_n)_{n=m}^\infty$ be convergent sequences of real numbers, and let $x,y$ be the real numbers $x:=\la$ and $y:=\lb$.
\begin{enumerate}[label=(\alph*)]
    \item  The sequence $(a_n+b_n)_{n=m}^\infty$ converges to $x+y$; in other words, $$\lim_{n\to\infty}(a_n+b_n)=\la+\lb.$$
    \item The sequence $(a_n b_n)_{n=m}^\infty$ converges to $xy$; in other words, $$\lim_{n\to\infty}(a_n b_n)=(\la)(\lb).$$
    \item For any real number $c$, the sequence $(ca_n)_{n=m}^\infty$ converges to $cx$; in other words, $$\lim_{n\to\infty}(ca_n)=c\la.$$
    \item The sequence $(a_n-b_n)_{n=m}^\infty$ converges to $x-y$; in other words, $$\lim_{n\to\infty}(a_n-b_n)=\la - \lb.$$
    \item Suppose that $y\ne 0$, and that $b_n\ne 0$ for all $n\geq m$. Then the sequence $(b_n^{-1})_{n=m}^\infty$ converges to $y^{-1}$; in other words,
    $$\lim_{n\to\infty}b_n^{-1}=(\lb)^{-1}.$$
    \item Suppose that $y\ne 0$, and that $b_n\ne 0$ for all $n\geq m$. Then the sequence $(a_n/b_n)_{n=m}^\infty$ converges to $x/y$; in other words,
    $$\lim_{n\to\infty}\frac{a_n}{b_n}=\frac{\la}{\lb}.$$
    \item The sequence $(\max(a_n,b_n))_{n=m}^\infty$ converges to $\max(x,y)$; in other words, $$\lim_{n\to\infty} \max(a_n,b_n)=\max(\la, \lb).$$
    \item The sequence $(\min(a_n,b_n))_{n=m}^\infty$ converges to $\min(x,y)$; in other words, $$\lim_{n\to\infty}\min(a_n,b_n)=\min (\la,\lb).$$
\end{enumerate}
\subsubsection*{Exercise 6.1.1.}
Let $(a_n)_{n=0}^\infty$ be a sequence of real numbers, such that $a_{n+1}>a_n$ for each natural number $n$. Prove that whenever $n$ and $m$ are natural numbers such that 
$m>n$, then we have $a_m>a_n$.
\begin{proof}
    Induct on $p$ to show that $a_{n+p}>a_n$ for all positive integers $p$. When $p=1$, as stated in the problem, we have $a_{n+p}=a_{n+1}>a_n$. Suppose inductively
    $a_{n+p}>a_n$ for positive integer $p$. Then $a_{n+p+1}>a_{n+p}>a_n$. This closes the induction. Since $m>n$, $m-n$ is a positive integer. Therefore, $a_{n+(m-n)}=a_m>a_n$.
\end{proof}
\subsubsection*{Exercise 6.1.2.}
Let $(a_n)_{n=m}^\infty$ be a sequence of real numbers, and let $L$ be a real number. Show that $\an$ converges to $L$ if and only if, given any real $\varepsilon>0$,
one can find an $N\geq m$ such that $|a_n-L|\leq \varepsilon$ for all $n\geq N$.
\begin{proof}
    Suppose $\an$ converges to $L$. By definition, $\an$ is eventually $\varepsilon$-close to $L$ for every real $\varepsilon>0$. Therefore, for any real $\varepsilon>0$,
    there exists $N\geq m$ such that $|a_n-L|\leq\varepsilon$ for all $n\geq N$.

    Suppose given any real $\varepsilon>0$, one can find an $N\geq m$ such that $|a_n-L|\leq \varepsilon$ for all $n\geq N$. Then any real $\varepsilon>0$, 
    there exists an $N\geq m$ such that $\an$ is eventually $\varepsilon$-close to $L$. Therefore, by definition, $\an$ converges to $L$.
\end{proof}
\subsubsection*{Exercise 6.1.3.}
Let $\an$ be sequence of real numbers, let $c$ be a real number, and let $m'\geq m$ be an integer. Show that $\an$ converges to $c$ if and only if $(a_n)_{n=m'}^\infty$ converges
to $c$.
\begin{proof}
    Suppose $\an$ converges to $c$. Then for any arbitrary $\varepsilon>0$, there exists $N_{\varepsilon}\geq m$ such that $|a_n-c|\leq \varepsilon$. So for any arbitrary $\varepsilon>0$,
    we can find $M_\varepsilon=\max(N_\varepsilon, m')$ such that $|a_n-c|\leq\varepsilon$ for all $n\geq M_\varepsilon\geq m'$. Therefore, $(a_n)_{n=m}^\infty$ converges to $c$.

    Suppose $(a_n)_{n=m'}^\infty$ converges to $c$. For any arbitrary $\varepsilon>0$, there exists $N_{\varepsilon}\geq m'$ such that $|a_n-c|\leq \varepsilon$ for all $n\geq N_\varepsilon$,
    since $m'\geq m$, we have $N_\varepsilon\geq m$.  Therefore, for any arbitrary $\varepsilon>0$, we can find $N_{\varepsilon}\geq m$ such that $|a_n-c|\leq \varepsilon$ for all $n\geq N_\varepsilon$.
    
    Thus, $\an$ converges to $c\iff (a_n)_{n=m}^\infty$ converges to $c$.
\end{proof}
\subsubsection*{Exercise 6.1.4.}
Let $\an$ be a sequence of real numbers, let $c$ be a real number,  and let $k\geq 0$ be a non-negative integer. Show that $\an$ converges to $c$
if and only if $(a_{n+k})_{n=m}^\infty$ converges to $c$.
\begin{proof}
    We can rewrite $(a_{n+k})_{n=m}^\infty=(a_n)_{m+k}^\infty$ since both of them are equal to the infinite sequence $a_{m+k},a_{m+k+1},\dotsc$. Since 
    $k\geq 0$, $m+k\geq m$. Therefore, by Exercise 6.1.4, we have $\an$ converges to $c$ if and only if $(a_n)_{m+k}^\infty=(a_{n+k})_{n=m}^\infty$ converges to $c$. 
\end{proof}
\subsubsection*{Exercise 6.1.5.}
Prove proposition 6.1.12.
\begin{proof}
    Suppose $\lim\limits_{n\to\infty}\an=L$. By definition, for all real $\varepsilon/2>0$, there exists $N_\varepsilon\geq m$ such that $|a_i-L|\leq\varepsilon/2$ and $|a_j-L|=|L-a_j|\leq\varepsilon/2$
    for all $i,j\geq N_\varepsilon$. Then for all real $\varepsilon>0$, let $N=N_\varepsilon\geq m$, we have $|a_i-a_j|\leq |a_i-L|+|L-a_j|\leq \varepsilon/2+\varepsilon/2=\varepsilon$ for all $n,m\geq N$.
    Therefore, $\an$ is a Cauchy sequence.
\end{proof}
\subsubsection*{Exercise 6.1.6.}
Prove Proposition 6.1.15.
\begin{proof}
    Write $L:=$LIM$_{n\to\infty}a_n$. We want to show that $L=\lim\limits_{n\to\infty}a_n$. Suppose $L$ and $\an$ are not eventually $\varepsilon$-close.
    Consider an arbitrary $\varepsilon/2>0$, since $\an$ is Cauchy, there exists $N\geq m$ such that $|a_i-a_j|\leq\varepsilon/2$ for all $i,j\geq N$. For this $N$, since $L$ and $\an$
    are not eventually $\varepsilon$-close, there exists $i\geq N$ such that $|a_i-L|>\varepsilon$. And since this $i\geq N$, we have $|a_i-a_j|\leq\varepsilon/2$ for all $j\geq i$.
    Since $|a_i-L|>\varepsilon$, we have 
    \begin{equation*}
        a_i > L+\varepsilon\text{ or }a_i < L-\varepsilon.
    \end{equation*}
    If $a_i>L+\varepsilon$, since $|a_i-a_j|\leq\varepsilon/2\implies a_i-\varepsilon/2\leq a_j\leq a_i+\varepsilon/2$, we have $L+\varepsilon/2<a_i-\varepsilon/2\leq a_j$.
    If $a_i<L-\varepsilon$, since $a_i-\varepsilon/2\leq a_j\leq a_i+\varepsilon/2$, we have $a_j\leq a_i+\varepsilon/2<L-\varepsilon/2$. Therefore, we either have $a_j<L-\varepsilon/2$
    or $a_j>L+\varepsilon/2$ for all $j\geq i$. 

    If $a_j<L-\varepsilon/2$ for all $j\geq i$, by Exericse 5.4.8, we have LIM$_{n\to\infty}a_n\leq L-\varepsilon/2<L$. If $a_j>L+\varepsilon/2$ for all $j\geq i$, by Exericse 5.4.8, we have LIM$_{n\to\infty}a_n\geq L+\varepsilon/2>L$.
    Then we have either LIM$_{n\to\infty}a_n>L$ or LIM$_{n\to\infty}a_n<L$, and it contradicts the fact that LIM$_{n\to\infty}a_n=L$.
    Thus, $L$ and $\an$ are eventually $\varepsilon$-close, hence $L=\lim\limits_{n\to\infty}a_n$.
\end{proof}
\subsubsection*{Exercise 6.1.7.}
Show that Definition 6.1.16 is consistent with Definition 5.1.12.
\begin{proof}
    We would like to show that if $(a_n)_{n=1}^\infty$ is bounded by $M$ then $\an$ is bounded by $M$. If $(a_n)_{n=1}^\infty$ is bounded by $M$, by Definition 5.1.12, 
    $|a_i|\leq M$ for all $i\geq 1$. Then for all $i\geq m\geq 1$, we have $|a_i|\leq M$. Therefore, $\an$ is bounded by $M$.
\end{proof}
\subsubsection*{Exercise 6.1.8.}
Prove Theorem 6.1.19.
\begin{enumerate}[label=(\alph*)]
    \item \begin{proof}
        Consider an arbitrary $\varepsilon>0$. Since $\an$ converges to $x$, there exists $N_1\geq m$ such that $|a_n-x|\leq\varepsilon/2$ for all $n\geq N_1$.
        Since $\bn$ converges to $x$, there exists $N_2\geq m$ such that $|b_n-x|\leq\varepsilon/2$ for all $n\geq N_2$. Let $N=\max(N_1, N_2)$, we have 
        $|(a_n+b_n)-(x+y)|=|(a_n-x)+(b_n-y)|\leq |a_n-x|+|b_n-y|\leq\varepsilon/2 +\varepsilon/2=\varepsilon$ for all $n\geq N$. Therefore, $\lim\limits_{a_n+b_n}=x+y$.
    \end{proof}
    \item \begin{proof}
        Consider an arbitrary $\varepsilon>0$. Since $\an$ converges to $x$, $\an$ is bounded by a positive number $M_1$. There exists $N_1\geq m$ such that $|a_n-x|\leq\varepsilon/2M_1$
        for all $n\geq N_1$. We can easily know that there exists a positive number $M_2$ such that $M_2\geq |x|$. Since $\bn$ converges to $y$, there exists $N_2$ such that 
        $|b_n-y|\leq \varepsilon/2M_2$ for all $n\geq N_2$. Let $N=\max(N_1,N_2)$, then 
        \begin{equation*}
            \begin{aligned}
                |a_n b_n-xy|&=|a_n b_n-xb_n+xb_n-xy|\\
                &\leq |b_n|\cdot |a_n-x|+|x|\cdot |b_n-y|\\
                &\leq M_1\cdot \frac{\varepsilon}{2M_1} + M_2\cdot \frac{\varepsilon}{2M_2}\\
                &=\frac{\varepsilon}{2}+\frac{\varepsilon}{2}\\
                &=\varepsilon.
            \end{aligned}
        \end{equation*}
        Therefore, $\lim\limits_{n\to\infty}(a_n b_n)=(\lim\limits_{n\to\infty}a_n)(\lim\limits_{n\to\infty}b_n)$.
    \end{proof}
    \item \begin{proof}
        Let $\bn=c,c,c,\dotsc$, by $(b)$, we have $\lim\limits_{n\to\infty}(a_n b_n)=\lim\limits_{n\to\infty}(c a_n)=c\lim\limits_{n\to\infty}(a_n)$.
    \end{proof}
    \item \begin{proof}
        By $(c)$, $\lim\limits_{n\to\infty}(-b_n)=-\lim\limits_{n\to\infty}b_n$. By $(a)$, we have $\lim\limits_{n\to\infty}(a_n-b_n)=\lim\limits_{n\to\infty}(a_n+(-b_n))
        =\lim\limits_{n\to\infty}a_n+\lim\limits_{n\to\infty}(-b_n)=\lim\limits_{n\to\infty}a_n-\lim\limits_{n\to\infty}(b_n)=x-y$. 
    \end{proof}
    \item \begin{proof}
        Since $\bn$ converges to $y$, there exists $N\geq m$ such that $|b_n-y|\leq|y|/2$ for all $n\geq N$. Then $y-|y|/2\leq b_n\leq y+|y|/2$. If $y<0$, $b_n\leq y+|y|/2=y-y/2=y/2=-|y|/2$.
        If $y>0$, $b_n\geq y-y/2=|y|/2$. Therefore, $|b_n|\geq |y|/2$ for $n\geq N$. Let $c=\min(|b_m|,\dotsc, |b_{N-1}|,|y|/2)$, we have $|b_i|\leq c$ for all $i\geq m$.
        So $\bn$ is a Cauchy sequence bounded away from zero, hence $(b_n^{-1})_{n=m}^\infty$ is also a Cauchy sequence.

        Consider an arbitrary $\varepsilon>0$. Since $|b_n|\geq c$ for all $n\geq m$, we have $1/|b_n|\leq 1/c$ for all $n\geq m$. Since $\bn$ converges to $y$,
        there exists $N\geq m$ such that $|b_n-y|\leq c|y|\varepsilon$. Then 
        \begin{equation*}
            \begin{aligned}
                \left|\frac{1}{b_n}-\frac{1}{y}\right|&=\left|\frac{b_n-y}{b_n\cdot y}\right|\\
                &=\frac{1}{|b_n|}\cdot \frac{1}{|y|}\cdot |b_n-y|\\
                &\leq \frac{1}{c}\cdot \frac{1}{|y|}\cdot c|y|\varepsilon\\
                &=\varepsilon.
            \end{aligned}
        \end{equation*}
        Thus, $(b_n^{-1})_{n=m}^\infty$ converges to $1/y$.
    \end{proof}
    \item \begin{proof}
        By $(b)$, $\lim\limits_{n\to\infty}\frac{a_n}{b_n}=(\lim\limits_{n\to\infty}a_n)(\lim\limits_{n\to\infty}\frac{1}{b_n})$. By $(e)$, $\lim\limits_{n\to\infty}\frac{1}{b_n}=1/y$.
        Therefore, $\lim\limits_{n\to\infty}\frac{a_n}{b_n}=x\cdot \frac{1}{y}=\frac{x}{y}$.
    \end{proof}
    \item \begin{proof} 
        Suppose $x\geq y$. Consider an arbitrary $\varepsilon>0$. Since $\an$ converges to $x$, there exists $N_1$ such that $|a_n-x|\leq (x-y)/2$ for $n\geq N_1$. Then $
        -(x-y)/2\leq a_n-x\leq (x-y)/2$, hence $a_n\geq (x+y)/2$. Since $\bn$ converges to $y$, there exists $N_2$ such that $|b_n-y|\leq (x-y)/2$ for $n\geq N_2$.
        Then $b_n\leq (x+y)/2$. Let $N=\max(N_1,N_2)$, we have $a_n\geq (x+y)/2\geq b_n$ for all $n\geq N$. So when $n\geq N$, $\max(a_n,b_n)=a_n$. Since $\an$ converges to $x$,
        there exists $M_1\geq m$ such that $|a_n-x|\leq \varepsilon$. Let $M=\max(N,M_1)$, we have $|\max(a_n,b_n)-\max(x,y)|=|a_n-x|\leq \varepsilon$ for all $n\geq M$. 

        Similarly, we can show that if $y>x$, there exists $M\geq m$ such that $|\max(a_n,b_n)-\max(x,y)|=|b_n-y|\leq \varepsilon$ for all $n\geq M$. Thus, $(\max(a_n,b_n))_{n=m}^\infty$ converges to $\max(x,y)$.
    \end{proof}
    \item \begin{proof}
        By $(b)$ and $(g)$, we have 
        \begin{equation*}
            \begin{aligned}
                \lim_{n\to\infty}\min(a_n,b_n)&=-\lim_{n\to\infty}\max(-a_n,-b_n)\\
                &=-\max(\lim_{n\to\infty}(-a_n),\lim_{n\to\infty}(-b_n))\\
                &=-\max(-\lim_{n\to\infty}(a_n),-\lim_{n\to\infty}(b_n))\\
                &=\min(\lim_{n\to\infty}a_n,\lim_{n\to\infty}b_n).
            \end{aligned}
        \end{equation*}
    \end{proof}
\end{enumerate}
\subsubsection*{Exercise 6.1.9.}
Explain why Theorem 6.1.19 (f) fails when the limit of the denominator is 0.
\begin{proof}
    0 does not have a reciprocal. A counterexample is $a_n=1$, $b_n=1/n$. Then $(a_n/b_n)_{n=m}^\infty=(n)_{n=m}^\infty$ is not convergent and hence does not have a limit. 
\end{proof}
\subsubsection*{Exercise 6.1.10.}
Show that the concept of equivalent Cauchy sequence, as defined in Definition 5.2.6, does not change if $\varepsilon$ is required to be positive real instead of positive rational.
\begin{proof}
    For all real $\varepsilon>0$, suppose $\an$ and $\bn$ are eventually $\varepsilon$-close. Since $\varepsilon$ is also a rational number, we have $\an$ and $\bn$ are eventually $\varepsilon$-close for every rational $\varepsilon>0$.
    Suppose $\an$ and $\bn$ are eventually $\varepsilon$-close for every rational $\varepsilon>0$. Consider an arbitrary real $\varepsilon>0$. Since there exists a rational number such that $0<\varepsilon'<\varepsilon$.
    Since $\an$ and $\bn$ are eventually $\varepsilon'$-close, there exists $N\geq m$ such that $|a_n-b_n|\leq\varepsilon'<\varepsilon$ for all $n\geq N$. Therefore, $\an$ and $\bn$
    are eventually $\varepsilon$-close for all real $\varepsilon>0$. 
\end{proof}
\subsection*{6.2 The Extended real number system}
\subsubsection*{Definition 6.2.1 (Extended real number system).}
The extended real number system $\mathbf{R^*}$ is the real line $\mathbf{R}$ with two additional elements attached, called $+\infty$ and $-\infty$. These elements are distinct from each 
other and also distinct from every real number. An extended real number $x$ is called finited iff it is a real number, and infinite iff it is equal to $+\infty$ and $-\infty$.
\subsubsection*{Definition 6.2.2 (Negation of extended reals).}
The operation of negation $x\mapsto -x$ on $\mathbf{R}$, we now extend to $\mathbf{R^*}$ by defining $-(+\infty):=-\infty$ and $-(-\infty):=+\infty$.
\subsubsection*{Definition 6.2.3 (Ordering of extended reals).}
Let $x$ and $y$ be extended real numbers. We say that $x\geq y$, i.e., $x$ is less than or equal to $y$, iff one of the following three statements is true:
\begin{enumerate}[label=(\alph*)]
    \item $x$ and $y$ are real numbers, and $x\geq y$ as real numbers.
    \item $y=+\infty$.
    \item $x=-\infty$.
\end{enumerate}
We say that $x<y$ if we have $x\leq y$ and $x\ne y$. We sometimes write $x<y$ as $y>x$, and $x\leq y$ as $y\geq x$. 
\subsubsection*{Proposition 6.2.5.}
Let $x,y,z$ be extended real numbers. Then the following statements are true:
\begin{enumerate}[label=(\alph*)]
    \item (Reflexivity) We have $x\leq x$.
    \item (Trichotomy) Exactly one of the statements $x<y$, $x=y$, or $x>y$ is true.
    \item (Transitivity) If $x\leq y$ and $y\leq z$, then $x\leq z$.
    \item (Negation reverses order) If $x\leq y$, then $-y\leq -x$.
\end{enumerate}
\subsubsection*{Definition 6.2.6 (Supremum of sets of extended reals).}
Let $E$ be a subset of $\mathbf{R^*}$. Then we define the supremum $\sup(E)$ or least upper bound of $E$ by the following rule. 
\begin{enumerate}[label=(\alph*)]
    \item If $E$ is contained in $\mathbf{R}$ (i.e., $+\infty$ and $-\infty$ are not elements of $E$), then we let $\sup(E)$ be as defined in Definition 5.5.10.
    \item If $E$ contains $+\infty$, then we set $\sup(E):=+\infty$.
    \item If $E$ does not contain $+\infty$ but does contain $-\infty$, then we set $\sup(E):=\sup(E\backslash\{-\infty\})$ (which is a subset of $\mathbf{R}$ and thus falls under case (a)).
\end{enumerate}
We also define the infimum $\inf(E)$ of $E$ (also known as the greatest lower bound of $E$) by the formula $$\inf(E):=-\sup(-E)$$ where $-E$ is the set $-E:=\{-x:x\in E\}$.
\subsubsection*{Theorem 6.2.11.}
Let $E$ be a subset of $\mathbf{R^*}$. Then the following statements are true.
\begin{enumerate}[label=(\alph*)]
    \item For every $x\in E$ we have $x\leq \sup(E)$ and $x\geq \inf(E)$.
    \item Suppose that $M\in\mathbf{R^*}$ is an upper bound for $E$, i.e., $x\leq M$ for all $x\in E$. Then we have $\sup(E)\leq M$.
    \item Suppose that $M\in\mathbf{R^*}$ is a lower bound for $E$, i.e., $x\geq M$ for all $x\in E$. Then we have $\inf(E)\geq M$. 
\end{enumerate}
\subsubsection*{Exercise 6.2.1.}
Prove Proposition 6.2.5.
\begin{enumerate}[label=(\alph*)]
    \item \begin{proof}
        If $x$ is a real number, since $x=x$, we have $x\leq x$. If $x=+\infty$, by Definition 6.2.3, we have $x\leq x$. If $x=-\infty$, by Definition 6.2.3, we have $x\leq x$.
        Therefore, $x\leq x$ for extended real $x$.
    \end{proof}
    \item \begin{proof}
        $x$ and $y$ are both real numbers. By Proposition 5.4.7, the statement is true. \\
        $x=+\infty$, $y=+\infty$. By Definition 6.2.3, we have $x\leq y$. Since $x=y$, $x\nless y$ and $y\nless x$. Similarly, we can show that if $x=-\infty$ and $y=-\infty$,
        we have $x=y$ but $x\nless y$ and $y\nless x$.\\
        $x=+\infty$, $y=-\infty$. Since $x=+\infty$, by Definition 6.2.3, we have $y\leq x$. As $x\ne y$. $y<x$. Since $y\ne +\infty$ and $x\ne -\infty$, $x\ne y$ does not hold.
        Thus, $x\nless y$.\\
        $x$ is real ($x=+\infty$), $y=+\infty$ ($y$ is real). Since $y=+\infty$, by Definition 6.2.3, $x\leq y$. Since $x\ne +\infty$, $x\ne$ and hence $x<y$. Also by Definition 6.2.3,
        $y\leq x$ does not hold, hence $y\nless x$. The cases of $x=-\infty$ ($x$ is real), $y$ is real ($y=-\infty$) can be shown in a similar way.
    \end{proof}
    \item \begin{proof}
        $x,y,z$ are real. Can be derived from the transitivity of real numbers.\\
        $x=-\infty$. By Definition 6.2.3, $x\leq z$.\\
        $y=-\infty$. By Definition 6.2.3, we have $x=-\infty$, hence $x\leq z$.\\
        $z=-\infty$. By Definition 6.2.3, $y=-\infty$. Since $x\leq y$, $x=-\infty$. Hence, $x\leq z$.\\
        $x=+\infty$. Since $x\leq y$, by Definition 6.2.3, we have $y=+\infty$. Since $y\leq z$, by Definition 6.2.3, we have $z=+\infty$. Hence, $x\leq z$.\\
        $y=+\infty$. Since $y\leq z$, by Definition 6.2.3, we have $z=+\infty$. Hence, $x\leq z$.\\
        $z=+\infty$. By Definition 6.2.3, $x\leq z$.
    \end{proof}
    \item \begin{proof}
        $x,y$ are real. Since $x\leq y$, $y-x\geq 0$. Then $-y-(-x)\leq 0$. Therefore, $-y\leq -x$.\\
        $y=+\infty$. Then $-y=-\infty$, by Definition 6.2.3, $-y\leq -x$.\\
        $x=-\infty$. Then $-x=+\infty$, by Definition 6.2.3, $-y\leq -x$.
    \end{proof}
\end{enumerate}
\subsubsection*{Exercise 6.2.2.}
Prove Theorem 6.2.11. 
\begin{enumerate}[label=(\alph*)]
    \item \begin{proof}
        $E\subseteq\mathbf{R}$. By the definition of supremum, infimum, and the least upper bound, we have $x\leq\sup(E)$ and $x\geq\inf(E)$.\\
        $+\infty\in E$, $-\infty\notin E$. By Definition 6.2.6, $\sup(E)=+\infty$, and by Definition 6.2.3, $x\leq\sup(E)$. In this case, $\inf(E)=\inf(E\backslash\{+\infty\})$. 
        For all $x\ne +\infty$, it follows the case we have proved above ($E\subseteq\mathbf{R}$). For $x=+\infty$, by Definition 6.2.3, $x\geq\inf(E)$.\\
        $-\infty\in E$, $+\infty\in E$. Similar to the case above.\\
        $+\infty\in E$, $-\infty\in E$. Since $+\infty\in E$, $x\leq\sup(E)=+\infty$. In this case, $\inf(E)=-\infty$. By Definition 6.2.3, $x\geq\inf(E)$.
    \end{proof}
    \item \begin{proof}
        $E\subseteq\mathbf{R}$. We have $x\leq\sup(E)\leq M$.\\
        $+\infty\in E$. Since $+\infty\leq M$, $M=+\infty$. By Definition 6.2.3, $\sup(E)\leq M=+\infty$.\\
        $-\infty\in E$, $+\infty\notin E$. $\sup(E)=\sup(E\backslash\{-\infty\})$, then $\sup(E)=\sup(E\backslash\{-\infty\})\leq M$.
    \end{proof}
    \item \begin{proof}
        For all $x\in E$, $M\leq x$. Then $-M\geq -x$ for all $-x\in -E$, so $-M\geq \sup(-E)$. Therefore, $M\leq -\sup(-E)=\inf(-E)$.  
    \end{proof}
\end{enumerate}
\end{document}