\documentclass[12pt, letter]{article}
\usepackage[utf8]{inputenc}
\usepackage[a4paper, total={6in, 8in}]{geometry}
\usepackage{tikz}
\usepackage[T1]{fontenc}
\usepackage{listings}
\usepackage{graphicx}
\usepackage{amsfonts}
\usepackage{amsmath}
\usepackage{amssymb}
\usepackage{enumitem}
\usepackage{amsthm}
\usepackage{mathtools}
\usepackage{listings}
\usepackage{bm}
\newcommand{\ssc}{\subsubsection* }
\newcommand{\an}{$(a_n)_{n=1}^\infty$ }
\newcommand{\bn}{$(b_n)_{n=1}^\infty$ }
\newcommand{\anz}{$(a_n)_{n=0}^\infty$ }
\newcommand{\bnz}{$(b_n)_{n=0}^\infty$ }
\newcommand{\cn}{$(c_n)_{n=1}^\infty$ }
\newcommand{\la}{LIM$_{n\to\infty}a_n$  }
\newcommand{\lb}{LIM$_{n\to\infty}b_n$}
\newcommand{\lc}{LIM$_{n\to\infty}c_n$}
\newcommand{\lao}{LIM$_{n\to\infty}a_n^{-1}$ }
\newcommand{\lbo}{LIM$_{n\to\infty}b_n^{-1}$ }
\newcommand{\lab}{LIM$_{n\to\infty}(a_n+b_n)$ }
\newcommand{\E}{Exercise }
\newcommand{\La}{[label=(\alph*)]}
\newcommand{\uvec}[1]{\boldsymbol{\hat{\textbf{#1}}}}
\usepackage[english]{babel}
\newtheorem{theorem}{Theorem}
\usepackage{setspace}

\setstretch{1.25}
\begin{document}
\section*{Chapter 5}
\section*{The real numbers}
\subsection*{5.1 Cauchy sequences}
\subsubsection*{Definition 5.1.1 (Sequences).}
Let $m$ be an integer. A sequence $(a_n)_{n=m}^{\infty}$ of rational numbers is any function from the set $\{n\in\mathbf{Z}:n\geq m\}$ to $\mathbf{Q}$, i.e.,
a mapping which assigns to each integer $n$ greater than or equal to $m$, a rational number $a_n$. More informally, a sequence $(a_n)_{n=m}^\infty$ of rational 
numbers is a collection of rationals $a_m$, $a_{m+1}$, $a_{m+2},\dotsc$
\subsubsection*{Definition 5.1.3 ($\varepsilon$-steadiness).}
Let $\varepsilon>0$. A sequence $(a_n)_{n=0}^\infty$ is said to be $\varepsilon$-steady iff each pair $a_j,a_k$ of sequence elements is $\varepsilon$-close for 
every natural number $j,k$. In other words, the sequence $a_0,a_1,a_2,\dotsc$ is $\varepsilon$-steady iff $|a_j-a_k|\leq\varepsilon$ for all $j,k$.
\subsubsection*{Definition 5.1.6 (Eventual $\varepsilon$-steadiness).}
Let $\varepsilon>0$. A sequence $(a_n)_{n=0}^\infty$ is said to be eventually $\varepsilon$-steady iff the sequence $a_N,a_{N+1},a_{N+2},\dotsc$ is $\varepsilon$-steady 
for some natural number $N\geq 0$. In other words, the sequence $a_0,a_1,a_2,\dotsc$ is eventually $\varepsilon$-steady iff there exists an $N\geq 0$ such that 
$|a_j-a_k|\leq\varepsilon$ for all $j,k\geq N$.
\subsubsection*{Definition 5.1.8 (Cauchy sequences).}
A sequence $(a_n)_{n=0}^\infty$ of rational numbers is said to be a Cauchy sequence iff for every rational $\varepsilon>0$, the sequence $(a_n)_{n=0}^\infty$ is 
eventually $\varepsilon$-steady. In other words, the sequence $a_0,a_1,a_2,\dotsc$ is a Cauchy sequence iff for every $\varepsilon>0$, there exists an $N\geq 0$
such that $d(a_j,a_k)$ for all $j,k\geq N$.
\subsubsection*{Proposition 5.1.11}
The sequence $a_1,a_2,a_3,\dotsc$ defined by $a_n:=1/n$ (i.e., the sequence $1,1/2,1/3,\dotsc$) is a Cauchy sequence. 
\subsubsection*{Definition 5.1.12 (Bounded sequences).} 
Let $M\geq 0$ be rational. A finite sequence $a_1,a_2,\dotsc,a_n$ is bounded by $M$ iff $|a_i|\leq M$ for all $1\leq i\leq n$. An infinite sequence $(a_n)_{n=1}^\infty$
is bounded by $M$ iff $|a_i|\leq M$ for all $i\geq 1$. A sequence is said to be bounded iff it is bounded by $M$ for some rational $M\geq 0$.
\subsubsection*{Lemma 5.1.14 (Finite sequences are bounded).}
Every finite sequence $a_1,a_2,\dotsc, a_n$ is bounded.
\subsubsection*{Lemma 5.1.15 (Cauchy sequences are bounded).}
Every Cauchy sequence $(a_n)_{n=1}^\infty$ is bounded.
\subsubsection*{Exercise 5.1.1}
Prove Lemma 5.1.15.
\begin{proof}
    Suppose $(a_n)_{n=1}^\infty$ is a Cauchy sequence. So for $\varepsilon=1$, there exists an $N\geq 1$ such that $d(a_j,a_k)\leq 1$ for all
    $j,k\geq N$. Then the sequence can be splite into two parts: $a_1,\dotsc,a_N$ and $a_{N+1},a_{N+2},\dotsc$. The former is a finite sequence, by Lemma 5.1.14, 
    it is bounded. Suppose this finite sequence is bounded by $M_1$. Consider $a_{N+1},a_{N+2},\dotsc$. Since it is $1$-steady, for any $i>N+1$, we have 
    $|a_i-a_{N+1}|\leq 1$. Rearrange the inequalities, we have $a_{N+1}-1\leq a_i\leq a_{N+1}+1$. Let $M_2=a_{N+1}+1$. Then $a_{N+1}<M_2$ and for every $i>N+1$, we have 
    $a_i\leq M_2$. So sequence $a_{N+1},a_{N+2},\dotsc$ is bounded by $M_2$. Let $M=\max\{M_1,M_2\}$. Then the Cauchy sequence $(a_n)_{n=1}^\infty$
    is bounded by $M$. 
\end{proof}
\subsection*{5.2 Equivalent Cauchy sequences}
\subsubsection*{Definition 5.2.1 ($\varepsilon$-close sequences).}
Let $(a_n)_{n=0}^\infty$ and $(b_n)_{n=0}^\infty$ be two sequences, and let $\varepsilon>0$. We say that the sequence $(a_n)_{n=0}^\infty$ is $\varepsilon$-close 
to $(b_n)_{n=0}^\infty$ iff $a_n$ is $\varepsilon$-close to $b_n$ for each $n\in\mathbf{N}$. In other words, the sequence $a_0,a_1,a_2,\dotsc$ is $\varepsilon$-close
to the sequence $b_1,b_1,b_2,\dotsc$ iff $|a_n-b_n|\leq\varepsilon$ for all $n=0,1,2,\dotsc$.
\subsubsection*{Definition 5.2.3 (Eventually $\varepsilon$-close sequences).}
Let $(a_n)_{n=0}^\infty$ and $(b_n)_{n=0}^\infty$ be two sequences, and let $\varepsilon>0$. We say that the sequence $(a_n)_{n=0}^\infty$ is eventually $\varepsilon$-close 
to $(b_n)_{n=0}^\infty$ iff there exists an $N\geq 0$ such that the sequences $(a_n)_{n=N}^\infty$ and $(b_n)_{n=N}^\infty$ are $\varepsilon$-close. In other words,
$a_0,a_1,a_2,\dotsc$ is eventually $\varepsilon$-close to $b_0,b_1,b_2,\dotsc$ iff there exists an $N\geq 0$ such that $|a_n-b_n|\leq\varepsilon$ for all $n\geq N$.
\subsubsection*{Definition 5.2.6 (Equivalent sequences).}
Two sequences $(a_n)_{n=0}^\infty$ and $(b_n)_{n=0}^\infty$ are equivalent iff for each rational $\varepsilon>0$, the sequences $(a_n)_{n=0}^\infty$ and 
$(b_n)_{n=0}^\infty$ are eventually $\varepsilon$-close. In other words, $a_0,a_1,a_2,\dotsc$ and $b_0,b_1,b_2,\dotsc$ are equivalent iff for every for every rational $\varepsilon>0$,
there exists an $N\geq 0$ such that $|a_n-b_n|\leq\varepsilon$ for all $n\geq N$.
\subsubsection*{Proposition 5.2.8}
Let $(a_n)_{n=1}^\infty$ and $(b_n)_{n=1}^\infty$ be the sequences $a_n=1+10^{-n}$ and $b_n=1-10^{-n}$. Then the sequences $a_n$, $b_n$ are equivalent.
\subsubsection*{Exercise 5.2.1}
Show that if $(a_n)_{n=1}^\infty$ and $(b_n)_{n=1}^\infty$ are equivalent sequences of rationals, then $(a_n)_{n=1}^\infty$ is a Cauchy sequence if and only if $(b_n)_{n=1}^\infty$
is a Cauchy sequence.
\begin{proof}
    Assume $(a_n)_{n=1}^\infty$ and $(b_n)_{n=1}^\infty$ are equivalent sequences of rationals, and $(a_n)_{n=1}^\infty$ is a Cauchy sequence. We want to show that for any rational 
    $\varepsilon>0$, there exists $N\geq 1$ such that $b_N,b_{N+1},\dotsc$ is $\varepsilon$-steady. 

    Since $(a_n)_{n=1}^\infty$ is a Cauchy sequence and $\frac{\varepsilon}{3}>0$, there exists $N_1\geq 1$ such that for all $i,j\geq N_1$, $|a_i-a_j|\leq \frac{\varepsilon}{3}$.
    Since $(a_n)_{n=1}^\infty$ and $(b_n)_{n=1}^\infty$ are equivalent, and $\frac{\varepsilon}{3}>0$, there exists $N_2\geq 1$ such that for all $i\geq N_2$, $|b_i-a_i|\leq\frac{\varepsilon}{3}$.
    Let $N=\max\{N_1,N_2\}$. Consider arbitrary $i,j\geq N$. Since $N\geq N_1$, we have 
    \begin{equation*}
        |a_i-a_j|\leq \frac{\varepsilon}{3}.
    \end{equation*}
    Since $N\geq N_2$, we have
    \begin{equation*}
        \begin{gathered}
            |b_i-a_i|\leq\frac{\varepsilon}{3}
        \end{gathered}
    \end{equation*}
    and 
    \begin{equation*}
        |a_j-b_j|\leq \frac{\varepsilon}{3}.
    \end{equation*}
    Since 
    \begin{equation*}
        |a_i-a_j|\leq \frac{\varepsilon}{3}
    \end{equation*}
    and 
    \begin{equation*}
        |b_i-a_i|\leq\frac{\varepsilon}{3},
    \end{equation*}
    we have 
    \begin{equation*}
        |b_i-a_j|\leq |a_i-a_j|+|b_i-a_i|\leq\frac{\varepsilon}{3}+\frac{\varepsilon}{3}=\frac{2\varepsilon}{3}.
    \end{equation*}
    Since 
    \begin{equation*}
        |a_j-b_j|\leq \frac{\varepsilon}{3}
    \end{equation*}
    and 
    \begin{equation*}
        |b_i-a_j|\leq\frac{2\varepsilon}{3},
    \end{equation*}
    we have 
    \begin{equation*}
        |b_i-b_j|\leq|a_j-b_j|+|b_i-a_j|\leq\frac{\varepsilon}{3}+\frac{2\varepsilon}{3}=\varepsilon.
    \end{equation*}
    Thus, for any $\varepsilon>0$, we can find $N=\max\{N_1,N_2\}$ such that $b_N,b_{N+1},dotsc$ is $\varepsilon$-steady. Therefore, $(b_n)_{i=1}^\infty$
    is a Cauchy sequence. 

    Similarly, we can show that if $(a_n)_{n=1}^\infty$ and $(b_n)_{n=1}^\infty$ are equivalent sequences of rationals and $(b_n)_{i=1}^\infty$ is 
    a Cauchy sequence, then $(a_n)_{n=1}^\infty$ is a Cauchy sequence. Thus, if $(a_n)_{n=1}^\infty$ and $(b_n)_{n=1}^\infty$ are equivalent sequences of rationals, then $(a_n)_{n=1}^\infty$ is a Cauchy sequence if and only if $(b_n)_{n=1}^\infty$
    is a Cauchy sequence.
\end{proof}
\subsubsection*{Exercise 5.2.2}
Let $\varepsilon>0$. Show that if $(a_n)_{n=1}^\infty$ and $(b_n)_{n=1}^\infty$ are eventually $\varepsilon$-close, then \an is bounded if and only if \bn is bounded.
\begin{proof}
    Suppose $(a_n)_{n=1}^\infty$ and $(b_n)_{n=1}^\infty$ are eventually $\varepsilon$-close and \an is bounded. Since $(a_n)_{n=1}^\infty$ and $(b_n)_{n=1}^\infty$ are eventually $\varepsilon$-close,
    for any $\varepsilon>0$, there exists $N\geq 1$ such that for any $i\geq N$, $|a_i-b_i|\leq\varepsilon$. Consider an arbitrary $\varepsilon>0$. We can find $N\geq 1$ such that for any 
    $i\geq N$, $|a_i-b_i|\leq \varepsilon$. Then 
    \begin{equation*}
        a_i-\varepsilon\leq b_i\leq a_i+\varepsilon.
    \end{equation*}
    Split \bn to $b_1,\dotsc,b_{N}$ and $b_{N+1},b_{N+2},\dotsc$. The former is a finite sequence, so it is bounded by some rational number $M_1$. Since \an is bounded, there exists $M$ such that $|a_i|\leq M$ for all $i\geq 1$. Then 
    \begin{equation*}
        -M\leq a_i\leq M.
    \end{equation*}
    So
    \begin{equation*}
        -M-\varepsilon\leq b_i\leq M+\varepsilon.
    \end{equation*}
    Therefore, $|b_i|\leq M+\varepsilon$ for all $i\geq N+1$. Let $M_0=\max(M,M_1)$. For any $i\geq 1$, we have $|b_i|\leq M_0$. Thus, \bn is bounded. 

    Similarly, we can show that if $(a_n)_{n=1}^\infty$ and $(b_n)_{n=1}^\infty$ are eventually $\varepsilon$-close, and \bn is bounded, then \an is bounded.

    Thus, if $(a_n)_{n=1}^\infty$ and $(b_n)_{n=1}^\infty$ are eventually $\varepsilon$-close, then \an is bounded if and only if \bn is bounded. 
\end{proof}
\subsection*{5.3 The construction of the real numbers}
\subsubsection*{Definition 5.3.1 (Real numbers).}
A real number is defined to be an object of the form \la, where \la is a Cauchy sequence of rational numbers. Two real numbers \la and \lb are said to be equal iff 
\an and \bn are equivalent Cauchy sequences. The set of all real numbers is denoted $\mathbf{R}$.
\subsubsection*{Proposition 5.3.3 (Formal limits are well-defined).}
Let $x=$ \la, $y=$ \lb, and $z=$ \lc be real numbers. Then, with the above definition of equality for real numbers, we have $x=x$. Also, if $x=y$, then $y=x$.
Finally, if $x=y$ and $y=z$, then $x=z$.
\subsubsection*{Definition 5.3.4 (Addition of reals).}
Let $x=$ \la and $y=$ \lb be real numbers. Then we define the sum $x+y$ to be $x+y:=$ \lab.
\subsubsection*{Lemma 5.3.6 (Sum of Cauchy sequences is Cauchy).}
Let $x=$ \la and $y=$ \lb be real numbers. Then $x+y$ is also a real number (i.e., $(a_n+b_n)_{n=1}^\infty$ is a Cauchy sequence of rationals).
\subsubsection*{Lemma 5.3.7 (Sums of equivalent Cauchy sequences are equivalent).}
Let $x=$ \la, $y=$ \lb, and $x'=$ LIM$_{n\to a_n'}$ be real numbers. Suppose that $x=x'$. Then we have $x+y=x'+y$.
\subsubsection*{Lemma 5.3.9 (Multiplication of reals).}
Let $x=$ \la and $y=$ \lb be real numbers. Then we define the product $xy$ to be $xy:=$ LIM$_{n\to\infty}a_n b_n$.
\subsubsection*{Proposition 5.3.10 (Multiplication is well defined).}
Let $x=$ \la, $y=$ \lb, and $x'=$ LIM$_{n\to\infty}a_n'$ be real numbers. Then $xy$ is also a real number. Furthermore, if $x=x'$, then $xy=x'y$.
\subsubsection*{Proposition 5.3.11}
All the laws of algebra from Proposition 4.1.6 hold not only for the integers, but for the reals as well.
\subsubsection*{Definition 5.3.12 (Sequences bounded away from zero).}
A sequence \an of rational numbers is said to be bounded away from zero iff there exists a raional number $c>0$ such that $|a_n|>c$ for all $n\geq 1$.
\subsubsection*{Lemma 5.3.14.}
Let $x$ be a non-zero real number. Then $x=$ \la for some Cauchy sequence \an which is bounded away from zero.
\subsubsection*{Lemma 5.3.15.}
Suppose that \an is a Cauchy sequence which is bounded away from zero. Then the sequence $(a_n^{-1})_{n=1}^\infty$ is also a Cauchy sequence.
\ssc{Definition 5.3.16 (Reciprocals of real numbers).}
Let $x$ be a non-zero real number. Let \an be a Cauchy sequence bounded away from zero such that $x=$ \la. Then we define the reciprocal $x^{-1}$ by the formula
$x^{-1}:=$ LIM$_{n\to\infty}a_n^{-1}$. 
\subsubsection*{Lemma 5.3.17 (Reciprocation is well defined).}
Let \an and \bn be two Cauchy sequences bounded away from zero such that \la = \lb (i.e., the two sequences are equivalent). Then \lao = \lbo.
\subsubsection*{Exercise 5.3.1}
Prove Proposition 5.3.3.
\begin{proof}
    Reflexivity. Since $x=$ \la, \an is a Cauchy sequence. Obviously, \an and \an are equivalent. Therefore, \la = \la ($x=x$).

    Symmetry. Assume $x=y$, then Cauchy sequences \an and \bn are equivalent. So for every $\varepsilon>0$, there exists $N\geq 1$ such that for every $i\geq N$, 
    $|a_i-b_i|=|b_i-a_i|\leq\varepsilon$. Therefore, \bn and \an are equivalent. Thus, \lb = \la ($y=x$).

    Transitivity. Assume $x=y$ and $y=z$. We want to show that the Cauchy sequences \an and \cn are equivalent, that is, 
    for any $\varepsilon>0$, there exists $N\geq 1$ such that for all $i\geq N$, we have $|a_i-c_i|\leq \varepsilon$. Suppose $\varepsilon$ is an arbitrary positive rational number.
     Since $x=y$, the Cauchy sequences \an and \bn are equivalent. Then there exists $N_1$ such that for every $i\geq N_1$, we have $|a_i-b_i|\leq \frac{\varepsilon}{2}$.
     Since $y=z$, the Cauchy sequences \bn and \cn are equivalent. Then there exists $N_2$ such that for every $i\geq N_2$, we have $|b_i-c_i|\leq \frac{\varepsilon}{2}$.
     Let $N=\max(N_1,N_2)$, then for all $i\geq N$, $|a_i-c_i|\leq |a_i-b_i|+|b_i-c_i|\leq \frac{\varepsilon}{2}+\frac{\varepsilon}{2}=\epsilon$. Thus, \an and \cn are equivalent. 
     So $x=z$.
\end{proof} 
\subsubsection*{Exercise 5.3.2}
Prove Proposition 5.3.10.
\begin{proof}
    $xy$ is a real number. We want to show that for any $\varepsilon>0$, there exists $N\geq 1$ such that $|a_i b_i-a_j b_j|\leq\varepsilon$ for any $i,j\geq N$. Consider an 
    arbitrary $\varepsilon>0$. Since \an and \bn are Cauchy sequences, by Lemma 5.1.15, they are both bounded. Assume \an is bounded by $M_1$ and \bn is bounded by $M_2$. Since $a_n$ is a Cauchy sequence, there exists $N_1\geq 1$ such that
    for all $i,j\geq N_1$, we have 
    \begin{equation*}
        |a_i-a_j|\leq \frac{\varepsilon}{2M_1}.
    \end{equation*} 
    Similarly, since \bn is a Cauchy sequence, there exists $N_2\geq 1$ such that for all $i,j\geq N_2$, we have 
    \begin{equation*}
        |b_i-b_j|\leq \frac{\varepsilon}{2M_2}.
    \end{equation*}
    Let $N=\max(N_1,N_2)$, consider an arbitrary pair of $i,j\geq N$. Then we have 
    \begin{equation*}
        \begin{aligned}
            |a_j b_j-a_i b_i|&=|a_j b_j-a_j b_i+a_j b_i-a_i b_i|\\
            &\leq |a_j b_j-a_j b_i|+|a_j b_i-a_i b_i|\\
            &=|a_j|\cdot |b_j-b_i|+|b_i|\cdot |a_j-a_i|\\
            &\leq M_1\cdot \frac{\varepsilon}{2M_1}+M_2\cdot\frac{\varepsilon}{2M_2}\\
            &=\frac{\varepsilon}{2}+\frac{\varepsilon}{2}\\
            &=\varepsilon.
        \end{aligned}
    \end{equation*}
    Therefore, for every $\varepsilon>0$, we can find an $N=\max(N_1,N_2)$ such that $a_N b_N,a_{N+1}b_{N+1},\dotsc$ is $\varepsilon$-close. Thus, $(a_n b_n)_{n=1}^\infty$
    is a Cauchy sequence and $xy$ is a real number.

    Since \bn is a Cauchy sequence, it must be bounded by some rational number $M$. Since \an and $(a_n')_{n=1}^\infty$ are equivalent, for every $\varepsilon>0$, there exists $N\geq 1$ such that 
    for all $i\geq N$, we have 
    \begin{equation*}
        |a_i-a_i'|\leq \frac{\varepsilon}{M}.
    \end{equation*}
    Therfore, for all $i\geq N$, 
    \begin{equation*}
        \begin{aligned}
            |a_i b_i-a_i' b_i|&=|b_i|\cdot |a_i-a_i'|\\
            &\leq M\cdot \frac{\varepsilon}{M}\\
            &=\varepsilon.
        \end{aligned}
    \end{equation*}
    Thus, $(a_n b_n)_{n=1}^\infty$ and $(a_n' b_n)_{n=1}^\infty$ are equivalent and that $xy=x'y$. 
\end{proof}
\subsubsection*{Exercise 5.3.3}
Let $a,b$ be rational numbers. Show that $a=b$ if and only if \la = \lb (i.e., the Cauchy sequences $a,a,a,a,\dotsc$ and $b,b,b,b,\dotsc$ are equivalent if and only if $a=b$).
This allows us to embed the rational numbers inside the real numbers in a well-defined manner.
\begin{proof}
    Suppose the Cauchy sequences $a,a,a,a,\dotsc$ and $b,b,b,b,\dotsc$ are equivalent. Assume $a\ne b$, then $|a-b|>0$. Since the two sequences are equivalent, for every $\varepsilon>0$,
    there exists $N\geq 1$ such that $|a_i-b_i|\leq \varepsilon$. Let $\varepsilon=\frac{|a-b|}{2}>0$. Then no matter what value $i$ is, we have 
    \begin{equation*}
        |a_i-b_i|=|a-b|>\frac{|a-b|}{2}=\frac{\varepsilon}{2}
    \end{equation*}
    which contradicts the definition of equivalent sequences. Therefore, $a=b$.

    Suppose $a=b$. Then for any $\varepsilon>0$, let $N=1$, we have 
    \begin{equation*}
        |a_i-b_i|=|a-b|=0<\varepsilon
    \end{equation*}
    for all $i\geq N$. Therefore, \an and \bn are equivalent.

    Thus, $a=b$ if and only if \la = \lb.
\end{proof}
\subsubsection*{Exercise 5.3.4}
Let \anz be a sequence of rational numbers which is bounded. Let \bnz be another sequence of rational numbers which is equivalent to \anz. 
Show that \bnz is also bounded.
\begin{proof}
    Suppose \anz is bounded by $M$. Similar to Exercise 5.2.2, we can split \bn to $b_0,\dotsc,b_N$ and $b_{N+1},b_{N+2},\dotsc$ such that the former is bounded by some rational number $M_1$ and 
    the latter is bounded by $M+\varepsilon$ for any $\varepsilon>0$. Let $M_0=\max(M,M_1)$, then \bnz is bounded by $M_0$. 
\end{proof}
\subsubsection*{Exercise 5.3.5}
Show that LIM$_{n\to\infty}1/n=0$. 
\begin{proof}
    We want to show that $a_n=1/n$ and $0,0,0,\dotsc$ are equivalent. Consider an arbitrary $\varepsilon>0$. Let $N=\lceil\frac{1}{\varepsilon}\rceil$, we have 
    \begin{equation*}
        |a_i-0|=a_i\leq a_N=\frac{1}{N}\leq \varepsilon
    \end{equation*}
    for all $i\geq N$. 
    Thus, LIM$_{n\to\infty}1/n=0$. 
\end{proof}
\subsection*{5.4 Ordering the reals}
\subsubsection*{Definition 5.4.1.}
Let \an be a sequence of rationals. We say that this sequence is positively bounded away from zero iff we have a positive rational $c>0$ such that $a_n\geq c$ for all $n\geq 1$ (in particular, 
the sequence is entirely positive). The sequence is negatively bounded away from zero iff we have a negative rational $-c<0$ such that $a_n\leq -c$ for all $n\geq 1$ (in particular, 
the sequence is entirely negative).
\subsubsection*{Definition 5.4.3.}
A real number $x$ is said to be positive iff it can be written as $x=$ \la for some Cauchy sequence \an which is positively bounded away from zero. $x$ is said to be 
negative iff it can be written as $x=$ \la for some sequence \an which is negatively bounded away from zero.
\subsubsection*{Proposition 5.4.4 (Basic properties of positive reals).}
For every real number $x$, exactly one of the following three statements is true: $(a)$ $x$ is zero; $(b)$ $x$ is positive; $(c)$ $x$ is negative. A real number $x$ is negative if and 
only if $-x$ is positive. If $x$ and $y$ are positive, then so are $x+y$ and $xy$.
\subsubsection*{Definition 5.4.5 (Absolute value).}
Let $x$ be a real number. We define the absolute value $|x|$ of $x$ to equal $x$ if $x$ is positive, $-x$ when $x$ is negative, and $0$ when $x$ is zero.
\subsubsection*{Definition 5.4.6 (Ordering of the real numbers).}
Let $x$ and $y$ be real numbers. We say that $x$ is greater than $y$, and write $x>y$, iff $x-y$ is a positive real number, and $x<y$ iff $x-y$ is a negative real number.
We define $x\geq y$ iff $x>y$ or $x=y$, and similarly define $x\leq y$.
\subsubsection*{Proposition 5.4.7.}
All the claims in Proposition 4.2.9 which held for rationals, continue to hold for real numbers.
\subsubsection*{Proposition 5.4.8.}
let $x$ be a positive real number. Then $x^{-1}$ is also positive. Also, if $y$ is another positive number and $x>y$, then $x^{-1}<y^{-1}$.
\subsubsection*{Proposition 5.4.9 (The non-negative reals are closed).}
Let $a_1,a_2,a_3,\dotsc$ be a Cauchy sequence of non-negative rational numbers. Then \la is a non-negative real number.
\subsubsection*{Corollary 5.4.10.}
Let \an and \bn be Cauchy sequences of rationals such that $a_n\geq b_n$ for all $n\geq 1$. Then \la  $\geq$ \lb.
\subsubsection*{Proposition 5.4.12 (Bounding of reals by rationals).}
Let $x$ be a positive real number. Then there exists a positive rational number $q$ such that $q\leq x$, and there exists a positive integer $N$
such that $x\leq N$.
\subsubsection*{Corollary 5.4.13 (Archimedean property).}
Let $x$ and $\varepsilon$ be any positive real numbers. Then there exists a positive integer $M$ such that $M\varepsilon>x$.
\subsubsection*{Proposition 5.4.14.}
Given any two real numbers $x<y$, we can find a rational number $q$ such that $x<q<y$.
\subsubsection*{Exercise 5.4.1.}
Prove Proposition 5.4.4.
\begin{proof}
    Assume $x=$ \la. \\
    At least one of the three statements is true. If \an is equivalent to $(0)_{n=1}^\infty$, $x$ is 0. If the Cauchy sequence \an is not equivalent to $(0)_{n=1}^\infty$, 
    by Lemma 5.3.14, \an is bounded away from zero. Then there exists a rational number $c>0$ such that $|a_i|\geq c$ for all $i\geq 1$. Since \an is a Cauchy sequence, 
    let $\varepsilon=c/2$, then there exists an there exists $N\geq 1$ such that $|a_i-a_j|\leq \varepsilon=c/2$ for all $i,j\geq N$. Let $j=N$, we have 
    $|a_i-a_N|\leq c/2$ for all $i\geq N$. Since \an is bounded away from 0 by $c$, $a_N$ cannot be $0$. If $a_N>0$, we have $a_N\geq c$ and $a_N-c/2\leq a_i\leq a_N+c/2$. So $a_i\geq a_N-c/2\geq c/2>0$, and 
    \an is eventually positively bounded away from zero. In particular, $a_i\geq c$ for all $i\geq N$. Let $b_i=c$ when $i< N$ and $b_i=a_i$ when $i\geq N$. Then \bn is equivalent to 
    \an and $x=$ \la $=$ \lb\,is a positive real number. Similarly, we can show that if $a_N<0$, $x$ would be negative. Thus, at least one of the three statements is true.

    At most one of the three statements is true. Suppose $x$ is zero. For any $c>0$, there exists $N\geq 1$ such that $|a_i-0|=|a_i|\leq \frac{c}{2}<c$. Therefore, \an is not bounded away from zero.
    Thus, $x$ is not positive nor negative. Suppose $x$ is positive. Then there exists $c>0$ such that $a_i>c>0$ for all $i\geq 1$. So for any $c'>0$, $a_i>0>-c'$. Therefore, $x$ cannot be negative. 
    Similarly, if $x$ is negative, it cannot be positive. Thus, at most one of the three statements is true.

    $x$ is negative $\iff -x$ is positive. We know that $-x=$ LIM$(-a_n)_{n=1}^\infty$. Suppose $x$ is negative. Then there exists $c>0$ such that $-a_i<-c$ for all $i\geq 1$. 
    So $a_i>c$ for all $i\geq 1$. Thus, \an is positively bounded away from zero, and $x=$ \la is positive. Similarly, we can show that if $-x$ is positive, then $x$ is negative.

    Assume $y=$ \lb. Suppose $x$ and $y$ are positive. Then there exist $c_1,c_2>0$ such that $a_i\geq c_1$ and $b_i\geq c_2$ for all $i\geq 1$. Let $c=c_1+c_2$, we have 
    $(a_i+b_i)\geq c=c_1+c_2$ for all $i\geq 1$. Therefore, $x+y$ is positive. Let $c'=c_1c_2$, we have $a_i b_i\geq c'=c_1c_2$ for all $i\geq 1$. Therefore, $xy$ is positive.
\end{proof}
\subsubsection*{Exercise 5.4.2.}
Prove the remaining claims in Proposition 5.4.7.
\begin{enumerate}[label=(\alph*)]
    \item \begin{proof}
        Since $x-y$ is a real number, by Proposition 5.4.4, exactly one of the three statements $x-y=0$, $x-y>0$, or $x-y<0$ is true. Thus, exactly one of $x=y$, $x>y$, or 
        $x<y$ is true.
    \end{proof}
    \item \begin{proof}
        Since $x-y$ is a real number, by Proposition 5.4.4, $x-y$ is negative iff $-(x-y)=y-x$ is positive. Thus, $x<y$ iff $y>x$.
    \end{proof}
    \item \begin{proof}
        Since $x<y$, we have $y>x$, hence, $y-x$ is positive. Similarly, since $y<z$, $z-y$ is positive. By Proposition 5.4.4, $z-x=(y-x)+(z-y)$ is positive.
        Therefore, $x<z$. 
    \end{proof}
    \item \begin{proof}
        Since $x<y$, we have $x-y=x-y+0=x-y+z-z=(x+z)-(y+z)<0$. Therefore, $x+z<y+z$.
    \end{proof}
\end{enumerate}
\subsubsection*{Exercise 5.4.3.}
Show that for every real number $x$ there is exactly one integer $N$ such that $N\leq x<N+1$. (This integer $N$ is called the integer part of $x$, and is sometimes denoted $N=\lfloor x\rfloor$.)
\begin{proof}
    Denote $x=$ \an where \an is a Cauchy sequence. Let $\varepsilon=\frac{1}{2}$. Then there exists an $N\geq 1$ such that $|x_i-x_N|\leq \varepsilon=\frac{1}{2}$ for all $i\geq N$. 
    Therefore, $|x-x_N|\leq \frac{1}{2}$. So 
    \begin{equation*}
        -\frac{1}{2}\leq x-x_N\leq \frac{1}{2}\implies x_N-\frac{1}{2}\leq x \leq x_N+\frac{1}{2}.
    \end{equation*}
    Since $x_N$ is rational, $x_N-\frac{1}{2}$ and $x_N+\frac{1}{2}$ are also rational. Then there exists exactly one integer $n$ such that 
    \begin{equation*}
        n\leq x_N-\frac{1}{2}<n+1
    \end{equation*}
    and 
    \begin{equation*}
        n+1\leq x_N+\frac{1}{2}<n+2.
    \end{equation*}
    Then there are two cases. If $x<n+1$, there exists exactly one integer $N$ such that 
    \begin{equation*}
        N=n\leq x_N-\frac{1}{2}\leq x<n+1=N+1.
    \end{equation*}
    If $x\geq n+1$, there exists exists exactly one integer $N$ such that 
    \begin{equation*}
        N=n+1\leq x\leq x_N+\frac{1}{2}<n+2=N+1.
    \end{equation*}
    Therefore, for every real number $x$ there is exactly one integer $N$ such that $N\leq x<N+1$.
 \end{proof}
 \subsubsection*{Exercise 5.4.4.}
Show that for any positive real number $x>0$ there exists a positive integer $N$ such that $x>1/N>0$.
\begin{proof}
    Let $x=$ \la. Since $x$ is a positive real number, it is positively bounded away from zero. Then there exists a positive rational number $c>0$, such that $a_n\geq c>0$
    for $n\geq 1$. By Corollary 5.4.10, we have $x=$ \la $\geq c>0$. Since $c>c/2>0$, we have $x\geq c>c/2>0$. Let $N=\frac{2}{c}+1$, then $0<\frac{1}{N}<\frac{c}{2}$. 
    Then we have $x>\frac{1}{N}>0$.
\end{proof}
\subsubsection*{Exercise 5.4.5.}
Prove Proposition 5.4.14.
\begin{proof}
    $y-x>0\implies y-x>0$. By Exercise 5.4.4, there exists a positive integer $N$ such that $y-x>1/N>0$. Then we have $Ny>Nx+1>Nx$. And by Exercise 5.4.3, there exists exactly one integer 
    $n$ such that $n\leq Nx<n+1$. Then we also have $n+1\leq Nx+1<n+2$. Therefore, $n\leq Nx<n+1\leq Nx+1<Ny$. Therefore, there exists an integer between $Nx$ and $Ny$. 
    Divide the inequalities by $N$, we have $x<\frac{n+1}{N}<y$ where $\frac{n+1}{N}$ is a rational number. Therefore, if $x<y$, we can find a rational number $q$ such that $x<q<y$.
\end{proof}
\subsubsection*{Exercise 5.4.6.}
Let $x,y$ be real numbers and let $\varepsilon>0$ be a positive real. Show that $|x-y|<\varepsilon$ if and only if $y-\varepsilon<x<y+\varepsilon$, and that $|x-y|\leq \varepsilon$ if and 
only if $y-\varepsilon\leq x\leq y+\varepsilon$.
\begin{itemize}
    \item $|x-y|<\varepsilon\iff y-\varepsilon<x<y+\varepsilon$.
    \begin{proof}
        Suppose $|x-y|<\varepsilon$. By definition, if $x-y>0$, we have $x-y<\varepsilon\implies x<y+\varepsilon$, and if $x-y<0$, we have $y-x<\varepsilon\implies y-\varepsilon<x$.
        Combining the two inequalities, we have $y-\varepsilon<x<y+\varepsilon$. 

        Suppose $y-\varepsilon<x<y+\varepsilon$. Then $-\varepsilon<x-y<\varepsilon$. So if $x-y$ is positive, $|x-y|=x-y<\varepsilon$. 
        Otherwise, $|x-y|=y-x<\varepsilon$. Therefore, $|x-y|<\varepsilon$.

        Thus, $|x-y|<\varepsilon\iff y-\varepsilon<x<y+\varepsilon$.
    \end{proof}
    \item $|x-y|\leq \varepsilon\iff y-\varepsilon\leq x\leq y+\varepsilon$.
    \begin{proof}
        The proof is almost identical to the previous one. 
    \end{proof}
\end{itemize}
\subsubsection*{Exercise 5.4.7.}
Let $x$ and $y$ be real numbers. Show that $x\leq y+\varepsilon$ for all real numbers $\varepsilon>0$ if and only if $x\leq y$. Show that $|x-y|\leq \varepsilon$
for all real numbers $\varepsilon>0$ if and only if $x=y$.
\begin{itemize}
    \item $x\leq y+\varepsilon$ for all $\varepsilon>0\iff x\leq y$.
    \begin{proof}
        Suppose $x\leq y+\varepsilon$ for all $\varepsilon>0$. Assume $x>y$. Then we have $x-y>\frac{x-y}{2}>0$. Let $\varepsilon=\frac{x-y}{2}$. We have 
        Then we have $x\leq y+\varepsilon=y+\frac{x-y}{2}\implies x\leq y$. (contradiction) Therefore, we must have $x\leq y$.

        Suppose $x\leq y$. For all $\varepsilon>0$, we have $x\leq y<y+\varepsilon$.

        Thus, $x\leq y+\varepsilon$ for all $\varepsilon>0\iff x\leq y$.
    \end{proof}
    \item $|x-y|\leq \varepsilon$ for all real numbers $\varepsilon>0 \iff x=y$.
    \begin{proof}
        Suppose $|x-y|\leq \varepsilon$ for all real numbers. By Exercise 5.4.6, we have $-\varepsilon<x-y<\varepsilon$. Since $x<y+\varepsilon$, we have $x\leq y$.
        Since $y<x+\varepsilon$, we have $y\leq x$. And since $y\leq x$ and $x\leq y$, we have $x=y$.

        Suppose $x=y$. Then $|x-y|=0\leq \varepsilon$ for all $\varepsilon>0$.

        Thus, $|x-y|\leq \varepsilon$ for all real numbers $\varepsilon>0 \iff x=y$.
    \end{proof}
\end{itemize}
\subsubsection*{Exercise 5.4.8.}
Let \an be a Cauchy sequence of rationals, and let $x$ be a real number. Show that if $a_n\leq x$ for all $n\geq 1$, then \la $\leq x$. Similarly, 
show that if $a_n\geq x$ for all $n\geq 1$, then \la $\geq x$.
\begin{proof}
    Suppose $a_n\leq x$ for all $n\geq 1$. Assume \la $> x$. Let $y=$ \la. If $y=$ \la $>x$, by Proposition 5.4.14, ther exists a rational number $q$
    such that $y=$ \la $> q>x$. On the other hand, we have $a_n\leq x<q$ for all $n\geq 1$. By Corollary 5.4.10, we have $y=$ \la $\leq q$ which contradicts 
    $y>q$. Therefore, \la $\leq x$. The second statement can be proved in a similar way.
\end{proof}
\end{document}